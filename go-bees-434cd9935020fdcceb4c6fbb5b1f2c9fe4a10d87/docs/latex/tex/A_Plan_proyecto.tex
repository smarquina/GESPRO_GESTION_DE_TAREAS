\apendice{Manuales}

\section{Introducción}\label{introduccion-plan}

La fase de planificación es un punto clave en cualquier proyecto. En
esta fase se estima el trabajo, el tiempo y el dinero que va a suponer
la realización del proyecto. Para ello, se analiza minuciosamente cada
una de las partes que componen el proyecto. Este análisis, además de
permitir conocer los recursos necesarios, es de gran ayuda en fases
posteriores del desarrollo. En este anexo se detalla todo este proceso.

La fase de planificación se puede dividir a su vez en:

\begin{itemize}
\tightlist
\item
  Planificación temporal.
\item
  Estudio de viabilidad.
\end{itemize}

En la primera parte, se elabora un calendario o un programa de tiempos.
En estos se estima el tiempo necesario para la realización de cada una
de las partes del proyecto. Se debe establecer una fecha fija de inicio
del proyecto y una fecha de finalización estimada. Teniendo en cuenta el
peso de cada una de las tareas y los requisitos que se deben cumplir
para poder empezar a trabajar en cada una de ellas.

La segunda parte se centra en la viabilidad del proyecto. El estudio de
viabilidad se puede dividir a su vez en dos apartados:

\begin{itemize}
\tightlist
\item
  \textbf{Viabilidad económica}: donde se estiman los costes y los beneficios que
  puede suponer la realización del proyecto.
\item
  \textbf{Viabilidad legal}: el contexto en el que se ejecuta el proyecto está
  regulado por una serie de leyes. Se deben analizar todas aquellas que
  afecten al proyecto. En el caso del software, las licencias y la Ley
  de Protección de Datos pueden ser los temas más relevantes.
\end{itemize}

\section{Planificación temporal}\label{planificacion-temporal}

Al inicio del proyecto se planteó utilizar una metodología ágil como
Scrum para la gestión del proyecto. Aunque no se ha seguido al 100\% la
metodología al tratarse de un proyecto educativo (no éramos un equipo de
4 a 8 personas, no hubo reuniones diarias, etc.), sí que se ha aplicado
en líneas generales una filosofía ágil:

\begin{itemize}
\tightlist
\item
  Se aplicó una estrategia de desarrollo incremental a través de
  iteraciones (\emph{sprints}) y revisiones.
\item
  La duración media de los \emph{sprints} fue de una semana.
\item
  Al finalizar cada \emph{sprint} se entregaba una parte del producto
  operativo (incremento).
\item
  Se realizaban reuniones de revisión al finalizar cada \emph{sprint} y
  al mismo tiempo de planificación del nuevo \emph{sprint}.
\item
  En la planificación del \emph{sprint} se generaba una pila de tareas a
  realizar.
\item
  Estas tareas se estimaban y priorizaban en un tablero \emph{canvas}.
\item
  Para monitorizar el progreso del proyecto se utilizó gráficos
  \emph{burndown}.
\end{itemize}

Comentar que la estimación se realizó mediante los \emph{story points}
que provee ZenHub y, a su vez, se les asignó una estimación temporal
(cota superior) que se recoge en la siguiente tabla:

\begin{longtable}[]{@{}ll@{}}
\toprule
\begin{minipage}[b]{0.22\columnwidth}\raggedright\strut
Story points\strut
\end{minipage} & \begin{minipage}[b]{0.31\columnwidth}\raggedright\strut
Estimación temporal\strut
\end{minipage}\tabularnewline
\midrule
\endhead
\begin{minipage}[t]{0.22\columnwidth}\raggedright\strut
1\strut
\end{minipage} & \begin{minipage}[t]{0.31\columnwidth}\raggedright\strut
15min\strut
\end{minipage}\tabularnewline
\begin{minipage}[t]{0.22\columnwidth}\raggedright\strut
2\strut
\end{minipage} & \begin{minipage}[t]{0.31\columnwidth}\raggedright\strut
45min\strut
\end{minipage}\tabularnewline
\begin{minipage}[t]{0.22\columnwidth}\raggedright\strut
3\strut
\end{minipage} & \begin{minipage}[t]{0.31\columnwidth}\raggedright\strut
2h\strut
\end{minipage}\tabularnewline
\begin{minipage}[t]{0.22\columnwidth}\raggedright\strut
5\strut
\end{minipage} & \begin{minipage}[t]{0.31\columnwidth}\raggedright\strut
5h\strut
\end{minipage}\tabularnewline
\begin{minipage}[t]{0.22\columnwidth}\raggedright\strut
8\strut
\end{minipage} & \begin{minipage}[t]{0.31\columnwidth}\raggedright\strut
12h\strut
\end{minipage}\tabularnewline
\begin{minipage}[t]{0.22\columnwidth}\raggedright\strut
13\strut
\end{minipage} & \begin{minipage}[t]{0.31\columnwidth}\raggedright\strut
24h\strut
\end{minipage}\tabularnewline
\begin{minipage}[t]{0.22\columnwidth}\raggedright\strut
21\strut
\end{minipage} & \begin{minipage}[t]{0.31\columnwidth}\raggedright\strut
2,5 días\strut
\end{minipage}\tabularnewline
\begin{minipage}[t]{0.22\columnwidth}\raggedright\strut
40\strut
\end{minipage} & \begin{minipage}[t]{0.31\columnwidth}\raggedright\strut
1 semana\strut
\end{minipage}\tabularnewline
\bottomrule
\caption{Equivalencia entre \emph{story points} y tiempo.}
\end{longtable}

A continuación se describen los diferentes \emph{sprints} que se han
realizado.

\subsection{Sprint 0 (09/09/16 -
16/09/16)}\label{sprint-0-090916---160916}

La reunión de planificación de este \emph{sprint} marcó el comienzo del
proyecto. En una reunión previa se había planteado la idea del proyecto
a Jose Francisco y este había aceptado tutorizarla. En esta nueva
reunión se profundizó en la idea, se incorporó Raúl Marticorena como
cotutor y se plantearon los objetivos del primer \emph{sprint}.

Los objetivos fueron: profundizar y formalizar los objetivos del
proyecto, investigar el estado del arte en algoritmos de detección y
tracking aplicados a la apicultura, establecer el conjunto de
herramientas que conformarían el entorno de desarrollo, la gestión del
proyecto y la comunicación del equipo y, por último, realizar un esquema
rápido de la aplicación que se deseaba desarrollar.

Las tareas en las que se descompusieron los objetivos se pueden ver en:
\href{https://github.com/davidmigloz/go-bees/milestone/1?closed=1}{Sprint
0}.

Se estimaron 8 horas de trabajo y se invirtieron finalmente 9,25 horas,
completando todas las tareas.

\imagen{burndowns/sprint0}{Sprint 0.}

\subsection{Sprint 1 (17/09/16 -
23/09/16)}\label{sprint-1-170916---230916}

Los objetivos de este \textbf{sprint} fueron: tomar contacto con OpenCV
para Android, realizar un curso \emph{online} de iniciación a Android,
investigar qué algoritmos de detección y tracking estaban disponibles en
OpenCV para Android y empezar a trabajar en la documentación.

En este \emph{sprint} se tuvo la suerte de hablar sobre el proyecto con
Rafael Saracchini, investigador en temas de visión artificial en el
Instituto Tecnológico de Castilla y León. Rafael nos propuso una serie
de algoritmos que nos podían ser útiles y otros que no funcionarían bajo
nuestros requisitos.

Las tareas en las que se descompusieron los objetivos se pueden ver en:
\href{https://github.com/davidmigloz/go-bees/milestone/2?closed=1}{Sprint
1}.

Se estimaron 7,25 horas de trabajo y se invirtieron finalmente 13,25
horas, completando todas las tareas.

\imagen{burndowns/sprint1}{Sprint 1.}

\subsection{Sprint 2 (24/09/16 -
29/09/16)}\label{sprint-2-240916---290916}

Los objetivos de este \emph{sprint} fueron: investigar cómo implementar
con OpenCV los algoritmos descritos en el \emph{sprint} anterior,
continuar la formación en Android y OpenCV y realizar grabaciones en el
colmenar para tener un conjunto de vídeos con los que realizar pruebas.

Las tareas en las que se descompusieron los objetivos se pueden ver en:
\href{https://github.com/davidmigloz/go-bees/milestone/3?closed=1}{Sprint
2}.

Mientras se realizaba una de las tareas del \emph{sprint}, se
encontraron dos \emph{bugs} relacionados con OpenCV y Android
(\href{https://github.com/davidmigloz/go-bees/issues/26}{\#26} y
\href{https://github.com/davidmigloz/go-bees/issues/27}{\#27}) que nos
impidieron continuar el desarrollo. El investigar su origen y buscar
soluciones supuso una gran cantidad de horas y no se lograron resolver
hasta el siguiente \emph{sprint}.

Se estimaron 11,75 horas de trabajo y se invirtieron finalmente 33
horas, quedando dos tareas pendientes para terminar durante el siguiente
\emph{sprint}.

\imagen{burndowns/sprint2}{Sprint 2.}

\subsection{Sprint 3 (30/09/16 -
06/10/16)}\label{sprint-3-300916---061016}

Los objetivos de este \emph{sprint} fueron: intentar resolver los bugs
descubiertos en el \emph{sprint} anterior, o si esto fuese imposible,
buscar una vía alternativa para continuar el proyecto y continuar
investigando las implementaciones de los algoritmos de extracción de
fondo en OpenCV.

Las tareas en las que se descompusieron los objetivos se pueden ver en:
\href{https://github.com/davidmigloz/go-bees/milestone/4?closed=1}{Sprint
3}.

Se estimaron 20,75 horas de trabajo y se invirtieron finalmente 31
horas, quedando una tarea por terminar.

\imagen{burndowns/sprint3}{Sprint 3.}

\subsection{Sprint 4 (07/10/16 -
13/10/16)}\label{sprint-4-071016---131016}

Los objetivos de este \emph{sprint} fueron: investigar técnicas de
preprocesado y potprocesado para mejorar los resultados de la fase de
extracción del fondo. Seleccionar y parametrizar el algoritmo de
extracción de fondo que provea los mejores resultados para nuestro
problema. Continuar el curso de Android. Integrar los servicios de
integración continua y documentación continua en el repositorio.

Las tareas en las que se descompusieron los objetivos se pueden ver en:
\href{https://github.com/davidmigloz/go-bees/milestone/5?closed=1}{Sprint
4}.

Se estimaron 37 horas de trabajo y se invirtieron finalmente 39,5 horas,
completando todas las tareas.

\imagen{burndowns/sprint4}{Sprint 4.}

\subsection{Sprint 5 (14/10/16 -
20/10/16)}\label{sprint-5-141016---201016}

Los objetivos de este \emph{sprint} fueron: afinar la parametrización de
los algoritmos implementados en el \emph{sprint} anterior. Detectar
contornos y contar los pertenecientes a abejas. Pensar algún método que
pueda solventar el problema del solapamiento de abejas. Documentar
\emph{sprint} anterior. Continuar la formación en Android.

Las tareas en las que se descompusieron los objetivos se pueden ver en:
\href{https://github.com/davidmigloz/go-bees/milestone/6?closed=1}{Sprint
5}.

Se estimaron 27 horas de trabajo y se invirtieron finalmente 34 horas,
completando todas las tareas.

\imagen{burndowns/sprint5}{Sprint 5.}

\subsection{Sprint 6 (21/10/16 -
27/10/16)}\label{sprint-6-211016---271016}

Los objetivos de este \emph{sprint} fueron: mudar el algoritmo de visión
artificial desarrollado en la plataforma Java a Android. Comenzar a
desarrollar una aplicación de testeo del algoritmo para conocer el error
que comete. Investigar si es posible simular el entorno de trabajo
filmando a una pantalla.

Las tareas en las que se descompusieron los objetivos se pueden ver en:
\href{https://github.com/davidmigloz/go-bees/milestone/7?closed=1}{Sprint
6}.

Mientras se mudaba el algoritmo a Android se encontró un \emph{bug} de
OpenCV (\href{https://github.com/davidmigloz/go-bees/issues/55}{\#55})
que agotaba la memoria del móvil. Este se debía a una mala liberación de
recursos por parte de OpenCV y resolvió liberándolos manualmente.

La tarea que más se desvió de su estimación fue la de testeo de los
algoritmos. Esto se debió a la dificultad añadida que supuso ejecutar
los test unitarios con dependencias de OpenCV en Travis. Finalmente, se
solventó instalando OpenCV en la máquina virtual de Travis (compilando
desde el código fuente) e inicializando la librería de forma estática
(ya que no se deseaba tener que arrancar un emulador para ejecutar los
tests unitarios).

Se estimaron 20,75 horas de trabajo y se invirtieron finalmente 41
horas, completando todas las tareas.

\imagen{burndowns/sprint6}{Sprint 6.}

\subsection{Sprint 7 (28/10/16 -
04/11/16)}\label{sprint-7-281016---041116}

Los objetivos de este \emph{sprint} fueron: estudiar patrón de
arquitectura MVP (\emph{Model-View-Presenter}) y pensar en cómo
aplicarlo al proyecto. Diseñar la posible arquitectura de la aplicación.
Estudiar el uso de inyección de dependencias en Android con Dagger 2.
Documentar las secciones de Introducción y Objetivos.

Las tareas en las que se descompusieron los objetivos se pueden ver en:
\href{https://github.com/davidmigloz/go-bees/milestone/8?closed=1}{Sprint
7}.

Se estimaron 16 horas de trabajo y se invirtieron finalmente 23 horas,
completando todas las tareas.

\imagen{burndowns/sprint7}{Sprint 7.}

\subsection{Sprint 8 (05/11/16 -
10/11/16)}\label{sprint-8-051116---101116}

Los objetivos de este \emph{sprint} fueron: diseñar el modelo de datos
de la aplicación teniendo en cuenta el uso final de estos. Desarrollar
una aplicación Java para realizar un conteo manual de un conjunto de
frames. Utilizar los datos obtenidos mediante la aplicación de conteo
para implementar un test que calcule el error que comete el algoritmo.

Las tareas en las que se descompusieron los objetivos se pueden ver en:
\href{https://github.com/davidmigloz/go-bees/milestone/9?closed=1}{Sprint
8}.

Se estimaron 46 horas de trabajo y se invirtieron finalmente 53 horas,
completando todas las tareas.

\imagen{burndowns/sprint8}{Sprint 8.}

\subsection{Sprint 9 (11/11/16 -
17/11/16)}\label{sprint-9-111116---171116}

Los objetivos de este \emph{sprint} fueron: implementar acceso a datos.
Inyección de dependencias con los \emph{build variants} de Gradle.
Empezar a desarrollar las distintas actividades de la app.

Las tareas en las que se descompusieron los objetivos se pueden ver en:
\href{https://github.com/davidmigloz/go-bees/milestone/10?closed=1}{Sprint
9}.

Se estimaron 23 horas de trabajo y se invirtieron finalmente 24,25
horas, completando todas las tareas.

\imagen{burndowns/sprint9}{Sprint 9.}

\subsection{Sprint 10 (11/11/16 -
17/11/16)}\label{sprint-10-111116---171116}

Los objetivos de este \emph{sprint} fueron: continuar desarrollando las
actividades principales de la app. Corregir documentación escrita hasta
el momento. Documentar Técnicas y herramientas y Aspectos relevantes.

Las tareas en las que se descompusieron los objetivos se pueden ver en:
\href{https://github.com/davidmigloz/go-bees/milestone/11?closed=1}{Sprint
10}.

Se estimaron 33,75 horas de trabajo y se invirtieron finalmente 39,25
horas, completando todas las tareas.

\imagen{burndowns/sprint10}{Sprint 10.}

\subsection{Sprint 11 (26/11/16 -
01/12/16)}\label{sprint-11-261116---011216}

Los objetivos de este \emph{sprint} fueron: implementar la vista detalle
de una colmena con sus grabaciones, pestañas en las vistas de colmenar y
colmena y la sección de ajustes. Corregir los errores en la
documentación indicados por los tutores. Continuar la formación en
Android.

Las tareas en las que se descompusieron los objetivos se pueden ver en:
\href{https://github.com/davidmigloz/go-bees/milestone/12?closed=1}{Sprint
11}.

Se estimaron 25,75 horas de trabajo y se invirtieron finalmente 34
horas, completando todas las tareas.

\imagen{burndowns/sprint11}{Sprint 11.}

\subsection{Sprint 12 (02/12/16 -
09/12/16)}\label{sprint-12-021216---091216}

Los objetivos de este \emph{sprint} fueron: implementar las partes de
visualización de los datos recogidos por la app (gráficos de actividad
de vuelo, temperatura, precipitaciones, vientes, etc.) Documentar
trabajos relacionados. Empezar a desarrollar la web del producto.

Las tareas en las que se descompusieron los objetivos se pueden ver en:
\href{https://github.com/davidmigloz/go-bees/milestone/13?closed=1}{Sprint
12}.

Se estimaron 36,25 horas de trabajo y se invirtieron finalmente 50,75
horas, completando todas las tareas.

\imagen{burndowns/sprint12}{Sprint 12.}

\subsection{Sprint 13 (10/12/16 -
14/12/16)}\label{sprint-13-101216---141216}

Los objetivos de este \emph{sprint} fueron: agregar opción de
localización GPS al añadir colmenar. Incluir una tabla comparativa en la
sección Trabajos relacionados.

Las tareas en las que se descompusieron los objetivos se pueden ver en:
\href{https://github.com/davidmigloz/go-bees/milestone/14?closed=1}{Sprint
13}.

Se estimaron 26,25 horas de trabajo y se invirtieron finalmente 14,25
horas, completando todas las tareas.

\imagen{burndowns/sprint13}{Sprint 13.}

\subsection{Sprint 14 (15/12/16 -
11/01/17)}\label{sprint-14-151216---110117}

Se trató del sprint más largo de todos los realizados, con una duración
de cuatro semanas debido a las vacaciones de Navidad.

Los objetivos de este \emph{sprint} fueron: implementar el servicio de
monitorización en segundo plano, junto con su sección de ajustes, la
obtención de información meteorológica, la edición y borrado de
colmenares y colmenas y las pestañas de información de colmenar y
colmena. Además, realizar un estudio de viabilidad legal y seleccionar
la licencia más apropiada para el proyecto.

Las tareas en las que se descompusieron los objetivos se pueden ver en:
\href{https://github.com/davidmigloz/go-bees/milestone/15?closed=1}{Sprint
14}.

Se estimaron 143 horas de trabajo y se invirtieron finalmente 187,75
horas, completando todas las tareas.

\imagen{burndowns/sprint14}{Sprint 14.}

\subsection{Sprint 15 (12/01/17 -
18/01/17)}\label{sprint-15-120117---180117}

Los objetivos de este \emph{sprint} fueron: finalizar el desarrollo
principal de la app completando el menú y la internacionalización.
Completar los contenidos de la memoria y continuar con los anexos ``Plan
del proyecto software'' y ``Requisitos.''

Las tareas en las que se descompusieron los objetivos se pueden ver en:
\href{https://github.com/davidmigloz/go-bees/milestone/16?closed=1}{Sprint
15}.

Se estimaron 39 horas de trabajo y se invirtieron finalmente 37,75
horas, a falta de terminar los anexos planificados por falta de tiempo.

\imagen{burndowns/sprint15}{Sprint 15.}

\subsection{Sprint 16 (19/01/17 -
25/01/17)}\label{sprint-16-190117---250117}

Los objetivos de este \emph{sprint} fueron: completar las tareas
pendientes del anterior sprint (Especificación de requisitos y Análisis
económico), documentar el diseño de datos, procedimental y
arquitectónico y aumentar la cobertura de los test.

Las tareas en las que se descompusieron los objetivos se pueden ver en:
\href{https://github.com/davidmigloz/go-bees/milestone/17?closed=1}{Sprint
16}.

Se estimaron 45,75 horas de trabajo y se invirtieron finalmente 45,25
horas, completando todas las tareas.

\imagen{burndowns/sprint16}{Sprint 16.}

\subsection{Sprint 17 (26/01/17 -
02/02/17)}\label{sprint-17-260117---020217}

Los objetivos de este \emph{sprint} fueron: continuar anexos. Convertir
la memoria a formato LaTeX. Pulir los últimos detalles de la aplicación
y publicarla en Google Play.

Las tareas en las que se descompusieron los objetivos se pueden ver en:
\href{https://github.com/davidmigloz/go-bees/milestone/18?closed=1}{Sprint
17}.

Se estimaron 53,50 horas de trabajo y se invirtieron finalmente 56,50
horas, completando todas las tareas.

\imagen{burndowns/sprint17}{Sprint 17.}

\subsection{Sprint 18 (02/02/17 -
07/02/17)}\label{sprint-18-020217---070217}

Los objetivos de este \emph{sprint} fueron: imprimir memoria, terminar
anexos y corrección de errores.

Las tareas en las que se descompusieron los objetivos se pueden ver en:
\href{https://github.com/davidmigloz/go-bees/milestone/19?closed=1}{Sprint
18}.

Se estimaron 41 horas de trabajo y se invirtieron finalmente 41 horas,
completando todas las tareas.

\imagen{burndowns/sprint18}{Sprint 18.}

\subsection{Resumen}\label{resumen}

En la siguiente tabla se muestra un resumen del tiempo dedicado a los
distintos tipos de tareas.

\begin{longtable}[]{@{}lrr@{}}
\toprule
\begin{minipage}[b]{0.37\columnwidth}\raggedright\strut
Categoría\strut
\end{minipage} & \begin{minipage}[b]{0.19\columnwidth}\raggedright\strut
\emph{Issues}\strut
\end{minipage} & \begin{minipage}[b]{0.19\columnwidth}\raggedright\strut
Tiempo (h)\strut
\end{minipage}\tabularnewline
\midrule
\endhead
\begin{minipage}[t]{0.37\columnwidth}\raggedright\strut
\emph{Bug}\strut
\end{minipage} & \begin{minipage}[t]{0.19\columnwidth}\raggedright\strut
26\strut
\end{minipage} & \begin{minipage}[t]{0.19\columnwidth}\raggedright\strut
40,75\strut
\end{minipage}\tabularnewline
\begin{minipage}[t]{0.37\columnwidth}\raggedright\strut
\emph{Documentation}\strut
\end{minipage} & \begin{minipage}[t]{0.19\columnwidth}\raggedright\strut
41\strut
\end{minipage} & \begin{minipage}[t]{0.19\columnwidth}\raggedright\strut
106\strut
\end{minipage}\tabularnewline
\begin{minipage}[t]{0.37\columnwidth}\raggedright\strut
\emph{Feature}\strut
\end{minipage} & \begin{minipage}[t]{0.19\columnwidth}\raggedright\strut
63\strut
\end{minipage} & \begin{minipage}[t]{0.19\columnwidth}\raggedright\strut
410\strut
\end{minipage}\tabularnewline
\begin{minipage}[t]{0.37\columnwidth}\raggedright\strut
\emph{Research}\strut
\end{minipage} & \begin{minipage}[t]{0.19\columnwidth}\raggedright\strut
30\strut
\end{minipage} & \begin{minipage}[t]{0.19\columnwidth}\raggedright\strut
128\strut
\end{minipage}\tabularnewline
\begin{minipage}[t]{0.37\columnwidth}\raggedright\strut
\emph{Testing}\strut
\end{minipage} & \begin{minipage}[t]{0.19\columnwidth}\raggedright\strut
7\strut
\end{minipage} & \begin{minipage}[t]{0.19\columnwidth}\raggedright\strut
49\strut
\end{minipage}\tabularnewline
\midrule
\begin{minipage}[t]{0.37\columnwidth}\raggedright\strut
TOTAL\strut
\end{minipage} & \begin{minipage}[t]{0.19\columnwidth}\raggedright\strut
167\strut
\end{minipage} & \begin{minipage}[t]{0.19\columnwidth}\raggedright\strut
794\strut
\end{minipage}\tabularnewline
\bottomrule
\caption{Desglose de las horas dedicadas al proyecto.}
\end{longtable}

\imagenAncho{project-sumary}{Porcentaje de horas dedicadas por categoría.}{0.7}
\newpage
\section{Estudio de viabilidad}\label{estudio-de-viabilidad}

\subsection{Viabilidad económica}\label{viabilidad-econuxf3mica}

En el siguiente apartado se analizarán los costes y beneficios que
podría haber supuesto el proyecto si se hubiese realizado en un entorno
empresarial real.

\subsubsection{Costes}\label{costes}

La estructura de costes del proyecto se puede desglosar en las
siguientes categorías.

\textbf{Costes de personal:}

El proyecto ha sido llevado a cabo por un desarrollador empleado a
tiempo completo durante cinco meses. Se considera el siguiente salario:

\begin{longtable}[]{@{}lr@{}}
\toprule
\begin{minipage}[b]{0.38\columnwidth}\raggedright\strut
\textbf{Concepto}\strut
\end{minipage} & \begin{minipage}[b]{0.20\columnwidth}\raggedright\strut
\textbf{Coste}\strut
\end{minipage}\tabularnewline
\midrule
\endhead
\begin{minipage}[t]{0.38\columnwidth}\raggedright\strut
Salario mensual neto\strut
\end{minipage} & \begin{minipage}[t]{0.20\columnwidth}\raggedright\strut
1.000\euro{}\strut
\end{minipage}\tabularnewline
\begin{minipage}[t]{0.38\columnwidth}\raggedright\strut
Retención IRPF (15\%)\strut
\end{minipage} & \begin{minipage}[t]{0.20\columnwidth}\raggedright\strut
272,23\euro{}\strut
\end{minipage}\tabularnewline
\begin{minipage}[t]{0.38\columnwidth}\raggedright\strut
Seguridad Social (29,9\%)\strut
\end{minipage} & \begin{minipage}[t]{0.20\columnwidth}\raggedright\strut
542,65\euro{}\strut
\end{minipage}\tabularnewline
\begin{minipage}[t]{0.38\columnwidth}\raggedright\strut
Salario mensual bruto\strut
\end{minipage} & \begin{minipage}[t]{0.20\columnwidth}\raggedright\strut
1.814,88\euro{}\strut
\end{minipage}\tabularnewline
\midrule
\begin{minipage}[t]{0.38\columnwidth}\raggedright\strut
\textbf{Total 5 meses}\strut
\end{minipage} & \begin{minipage}[t]{0.20\columnwidth}\raggedright\strut
9.074,40 \euro{}\strut
\end{minipage}\tabularnewline
\bottomrule
\caption{Costes de personal.}
\end{longtable}

La retribución a la Seguridad Social se ha calculado como un 23,60\% por
contingencias comunes, más un 5,50\% por desempleo de tipo general, más
un 0,20\% para el Fondo de Garantía Salarial y más un 0,60\% de
formación profesional. En total un 29,9\% que se aplica al salario bruto
\citep{ss_cotizacion}.

\textbf{Costes de \emph{hardware}:}

En este apartado se revisan todos los costes en dispositivos
\emph{hardware} que se han necesitado para el desarrollo del proyecto.
Se considera que la amortización ronda los 5 años y han sido utilizados
durante 5 meses.

\begin{longtable}[]{@{}lrr@{}}
\toprule
\begin{minipage}[b]{0.29\columnwidth}\raggedright\strut
\textbf{Concepto}\strut
\end{minipage} & \begin{minipage}[b]{0.18\columnwidth}\raggedright\strut
\textbf{Coste}\strut
\end{minipage} & \begin{minipage}[b]{0.32\columnwidth}\raggedright\strut
\textbf{Coste amortizado}\strut
\end{minipage}\tabularnewline
\midrule
\endhead
\begin{minipage}[t]{0.29\columnwidth}\raggedright\strut
Dispositivo móvil\strut
\end{minipage} & \begin{minipage}[t]{0.18\columnwidth}\raggedright\strut
300\euro{}\strut
\end{minipage} & \begin{minipage}[t]{0.32\columnwidth}\raggedright\strut
25\euro{}\strut
\end{minipage}\tabularnewline
\begin{minipage}[t]{0.29\columnwidth}\raggedright\strut
Ordenador portátil\strut
\end{minipage} & \begin{minipage}[t]{0.18\columnwidth}\raggedright\strut
800\euro{}\strut
\end{minipage} & \begin{minipage}[t]{0.32\columnwidth}\raggedright\strut
66,67\euro{}\strut
\end{minipage}\tabularnewline
\midrule
\begin{minipage}[t]{0.29\columnwidth}\raggedright\strut
\textbf{Total}\strut
\end{minipage} & \begin{minipage}[t]{0.18\columnwidth}\raggedright\strut
1.100\euro{}\strut
\end{minipage} & \begin{minipage}[t]{0.32\columnwidth}\raggedright\strut
91,67\euro{}\strut
\end{minipage}\tabularnewline
\bottomrule
\caption{Costes de \emph{hardware}.}
\end{longtable}
\newpage
\textbf{Costes de \emph{software}:}

En este apartado se revisan todos los costes en licencias de
\emph{software} no gratuito. Se considera que la amortización del
\emph{software} ronda los 2 años.

\begin{longtable}[]{@{}lrr@{}}
\toprule
\begin{minipage}[b]{0.24\columnwidth}\raggedright\strut
\textbf{Concepto}\strut
\end{minipage} & \begin{minipage}[b]{0.18\columnwidth}\raggedright\strut
\textbf{Coste}\strut
\end{minipage} & \begin{minipage}[b]{0.32\columnwidth}\raggedright\strut
\textbf{Coste amortizado}\strut
\end{minipage}\tabularnewline
\midrule
\endhead
\begin{minipage}[t]{0.24\columnwidth}\raggedright\strut
Windows 10 Pro\strut
\end{minipage} & \begin{minipage}[t]{0.18\columnwidth}\raggedright\strut
279\euro{}\strut
\end{minipage} & \begin{minipage}[t]{0.32\columnwidth}\raggedright\strut
58,13\euro{}\strut
\end{minipage}\tabularnewline
\begin{minipage}[t]{0.24\columnwidth}\raggedright\strut
Creately\strut
\end{minipage} & \begin{minipage}[t]{0.18\columnwidth}\raggedright\strut
5\euro{}\strut
\end{minipage} & \begin{minipage}[t]{0.32\columnwidth}\raggedright\strut
1,04\euro{}\strut
\end{minipage}\tabularnewline
\midrule
\begin{minipage}[t]{0.24\columnwidth}\raggedright\strut
\textbf{Total}\strut
\end{minipage} & \begin{minipage}[t]{0.18\columnwidth}\raggedright\strut
284\euro{}\strut
\end{minipage} & \begin{minipage}[t]{0.32\columnwidth}\raggedright\strut
59,17\euro{}\strut
\end{minipage}\tabularnewline
\bottomrule
\caption{Costes de \emph{software}.}
\end{longtable}

\textbf{Costes varios:}

En este apartado se revisan el resto de costes del proyecto.

\begin{longtable}[]{@{}lr@{}}
\toprule
\begin{minipage}[b]{0.48\columnwidth}\raggedright\strut
\textbf{Concepto}\strut
\end{minipage} & \begin{minipage}[b]{0.18\columnwidth}\raggedright\strut
\textbf{Coste}\strut
\end{minipage}\tabularnewline
\midrule
\endhead
\begin{minipage}[t]{0.48\columnwidth}\raggedright\strut
Dominio gobees.io\strut
\end{minipage} & \begin{minipage}[t]{0.18\columnwidth}\raggedright\strut
31,90\euro{}\strut
\end{minipage}\tabularnewline
\begin{minipage}[t]{0.48\columnwidth}\raggedright\strut
Cuenta Google Play\strut
\end{minipage} & \begin{minipage}[t]{0.18\columnwidth}\raggedright\strut
25\euro{}\strut
\end{minipage}\tabularnewline
\begin{minipage}[t]{0.48\columnwidth}\raggedright\strut
Memoria impresa y cartel\strut
\end{minipage} & \begin{minipage}[t]{0.18\columnwidth}\raggedright\strut
50\euro{}\strut
\end{minipage}\tabularnewline
\begin{minipage}[t]{0.48\columnwidth}\raggedright\strut
Alquiler de oficina\strut
\end{minipage} & \begin{minipage}[t]{0.18\columnwidth}\raggedright\strut
500\euro{}\strut
\end{minipage}\tabularnewline
\begin{minipage}[t]{0.48\columnwidth}\raggedright\strut
Internet\strut
\end{minipage} & \begin{minipage}[t]{0.18\columnwidth}\raggedright\strut
150\euro{}\strut
\end{minipage}\tabularnewline
\begin{minipage}[t]{0.48\columnwidth}\raggedright\strut
Material de apicultura de prueba\strut
\end{minipage} & \begin{minipage}[t]{0.18\columnwidth}\raggedright\strut
150\euro{}\strut
\end{minipage}\tabularnewline
\midrule
\begin{minipage}[t]{0.48\columnwidth}\raggedright\strut
\textbf{Total}\strut
\end{minipage} & \begin{minipage}[t]{0.18\columnwidth}\raggedright\strut
906,90\euro{}\strut
\end{minipage}\tabularnewline
\bottomrule
\caption{Costes varios.}
\end{longtable}

\textbf{Costes totales:}

El sumatorio de todos los costes es el siguiente:

\begin{longtable}[]{@{}lr@{}}
\toprule
\begin{minipage}[b]{0.22\columnwidth}\raggedright\strut
\textbf{Concepto}\strut
\end{minipage} & \begin{minipage}[b]{0.22\columnwidth}\raggedright\strut
\textbf{Coste}\strut
\end{minipage}\tabularnewline
\midrule
\endhead
\begin{minipage}[t]{0.22\columnwidth}\raggedright\strut
Personal\strut
\end{minipage} & \begin{minipage}[t]{0.22\columnwidth}\raggedright\strut
9.074,40\euro{}\strut
\end{minipage}\tabularnewline
\begin{minipage}[t]{0.22\columnwidth}\raggedright\strut
\emph{Hardware}\strut
\end{minipage} & \begin{minipage}[t]{0.22\columnwidth}\raggedright\strut
91,67\euro{}\strut
\end{minipage}\tabularnewline
\begin{minipage}[t]{0.22\columnwidth}\raggedright\strut
\emph{Software}\strut
\end{minipage} & \begin{minipage}[t]{0.22\columnwidth}\raggedright\strut
59,17\euro{}\strut
\end{minipage}\tabularnewline
\begin{minipage}[t]{0.22\columnwidth}\raggedright\strut
Varios\strut
\end{minipage} & \begin{minipage}[t]{0.22\columnwidth}\raggedright\strut
906,90\euro{}\strut
\end{minipage}\tabularnewline
\midrule
\begin{minipage}[t]{0.22\columnwidth}\raggedright\strut
Total\strut
\end{minipage} & \begin{minipage}[t]{0.22\columnwidth}\raggedright\strut
10.132,14\euro{}\strut
\end{minipage}\tabularnewline
\bottomrule
\caption{Costes totales.}
\end{longtable}

\subsubsection{Beneficios}\label{beneficios}

La aplicación desarrollada se distribuirá de forma gratuita y sin
publicidad, por lo que a corto plazo no se obtendrán beneficios.

La forma de monetizar la aplicación será en una segunda fase, cuando se
desarrolle una plataforma en la nube que sincronice la información de
varios dispositivos y permita el acceso remoto a la información.

Se considerarán tres tipos de suscripciones:

\begin{longtable}[]{@{}lllll@{}}
\toprule
\begin{minipage}[b]{0.16\columnwidth}\raggedright\strut
\textbf{Tipo}\strut
\end{minipage} & \begin{minipage}[b]{0.19\columnwidth}\raggedright\strut
\textbf{Colmenares}\strut
\end{minipage} & \begin{minipage}[b]{0.17\columnwidth}\raggedright\strut
\textbf{Colmenas}\strut
\end{minipage} & \begin{minipage}[b]{0.20\columnwidth}\raggedright\strut
\textbf{Plataformas}\strut
\end{minipage} & \begin{minipage}[b]{0.15\columnwidth}\raggedright\strut
\textbf{Precio}\strut
\end{minipage}\tabularnewline
\midrule
\endhead
\begin{minipage}[t]{0.16\columnwidth}\raggedright\strut
Hobby\strut
\end{minipage} & \begin{minipage}[t]{0.19\columnwidth}\raggedright\strut
1\strut
\end{minipage} & \begin{minipage}[t]{0.17\columnwidth}\raggedright\strut
10\strut
\end{minipage} & \begin{minipage}[t]{0.20\columnwidth}\raggedright\strut
App / Cloud\strut
\end{minipage} & \begin{minipage}[t]{0.15\columnwidth}\raggedright\strut
Gratis\strut
\end{minipage}\tabularnewline
\begin{minipage}[t]{0.16\columnwidth}\raggedright\strut
Amateur\strut
\end{minipage} & \begin{minipage}[t]{0.19\columnwidth}\raggedright\strut
5\strut
\end{minipage} & \begin{minipage}[t]{0.17\columnwidth}\raggedright\strut
100\strut
\end{minipage} & \begin{minipage}[t]{0.20\columnwidth}\raggedright\strut
App / Cloud\strut
\end{minipage} & \begin{minipage}[t]{0.15\columnwidth}\raggedright\strut
5\euro{}/mes\strut
\end{minipage}\tabularnewline
\begin{minipage}[t]{0.16\columnwidth}\raggedright\strut
Profesional\strut
\end{minipage} & \begin{minipage}[t]{0.19\columnwidth}\raggedright\strut
Ilimitados\strut
\end{minipage} & \begin{minipage}[t]{0.17\columnwidth}\raggedright\strut
Ilimitados\strut
\end{minipage} & \begin{minipage}[t]{0.20\columnwidth}\raggedright\strut
App / Cloud\strut
\end{minipage} & \begin{minipage}[t]{0.15\columnwidth}\raggedright\strut
20\euro{}/mes\strut
\end{minipage}\tabularnewline
\bottomrule
\caption{Tipos de suscripciones.}
\end{longtable}

\subsection{Viabilidad legal}\label{viabilidad-legal}

En esta sección se discutirán los temas relacionados con las licencias.
Tanto del propio \emph{software}, como de su documentación, imágenes y
vídeos.

``En Derecho, una licencia es un contrato mediante el cual una persona
recibe de otra el derecho de uso, de copia, de distribución, de estudio
y de modificación (en el caso del \emph{Software} Libre) de varios de
sus bienes, normalmente de carácter no tangible o intelectual, pudiendo
darse a cambio del pago de un monto determinado por el uso de los
mismos.'' \citep{wiki:licencia}

\subsubsection{Software}\label{software}

En primer lugar, vamos a analizar cuál sería la licencia más conveniente
para nuestro proyecto. Por un lado, somos nosotros los que podemos
elegir qué derechos queremos proporcionar a los usuarios y cuáles no.
Sin embargo, estamos limitados por los derechos que nos conceden a
nosotros las licencias de las dependencias utilizadas en el proyecto.

A continuación, se muestran las licencias de las dependencias usadas.

\begin{longtable}[]{@{}llll@{}}
\toprule
\begin{minipage}[b]{0.18\columnwidth}\raggedright\strut
Dependencia\strut
\end{minipage} & \begin{minipage}[b]{0.10\columnwidth}\raggedright\strut
Versión\strut
\end{minipage} & \begin{minipage}[b]{0.49\columnwidth}\raggedright\strut
Descripción\strut
\end{minipage} & \begin{minipage}[b]{0.11\columnwidth}\raggedright\strut
Licencia\strut
\end{minipage}\tabularnewline
\midrule
\endhead
\begin{minipage}[t]{0.18\columnwidth}\raggedright\strut
Android Support Library\strut
\end{minipage} & \begin{minipage}[t]{0.08\columnwidth}\raggedright\strut
25.1.0\strut
\end{minipage} & \begin{minipage}[t]{0.49\columnwidth}\raggedright\strut
Biblioteca de compatibilidad de Android.\strut
\end{minipage} & \begin{minipage}[t]{0.11\columnwidth}\raggedright\strut
Apache v2.0\strut
\end{minipage}\tabularnewline
\begin{minipage}[t]{0.18\columnwidth}\raggedright\strut
OpenCV\strut
\end{minipage} & \begin{minipage}[t]{0.08\columnwidth}\raggedright\strut
3.1.0\strut
\end{minipage} & \begin{minipage}[t]{0.49\columnwidth}\raggedright\strut
Biblioteca de visión artificial.\strut
\end{minipage} & \begin{minipage}[t]{0.11\columnwidth}\raggedright\strut
BSD\strut
\end{minipage}\tabularnewline
\begin{minipage}[t]{0.18\columnwidth}\raggedright\strut
Google Play Services\strut
\end{minipage} & \begin{minipage}[t]{0.08\columnwidth}\raggedright\strut
10.0.1\strut
\end{minipage} & \begin{minipage}[t]{0.49\columnwidth}\raggedright\strut
Biblioteca que proporciona acceso a diferentes servicios, entre ellos,
localización.\strut
\end{minipage} & \begin{minipage}[t]{0.11\columnwidth}\raggedright\strut
Apache v2.0\strut
\end{minipage}\tabularnewline
\begin{minipage}[t]{0.18\columnwidth}\raggedright\strut
Guava\strut
\end{minipage} & \begin{minipage}[t]{0.08\columnwidth}\raggedright\strut
20.0\strut
\end{minipage} & \begin{minipage}[t]{0.49\columnwidth}\raggedright\strut
Conjunto de bibliotecas comunes para Java.\strut
\end{minipage} & \begin{minipage}[t]{0.11\columnwidth}\raggedright\strut
Apache v2.0\strut
\end{minipage}\tabularnewline
\begin{minipage}[t]{0.18\columnwidth}\raggedright\strut
RoundedImage\strut
\end{minipage} & \begin{minipage}[t]{0.08\columnwidth}\raggedright\strut
2.3.0\strut
\end{minipage} & \begin{minipage}[t]{0.49\columnwidth}\raggedright\strut
Componente para mostrar imágenes redondeadas en Android.\strut
\end{minipage} & \begin{minipage}[t]{0.11\columnwidth}\raggedright\strut
Apache v2.0\strut
\end{minipage}\tabularnewline
\begin{minipage}[t]{0.18\columnwidth}\raggedright\strut
MPChart\strut
\end{minipage} & \begin{minipage}[t]{0.08\columnwidth}\raggedright\strut
3.0.1\strut
\end{minipage} & \begin{minipage}[t]{0.49\columnwidth}\raggedright\strut
Biblioteca de gráficos para Android.\strut
\end{minipage} & \begin{minipage}[t]{0.11\columnwidth}\raggedright\strut
Apache v2.0\strut
\end{minipage}\tabularnewline
\begin{minipage}[t]{0.18\columnwidth}\raggedright\strut
VNTPicker Preference\strut
\end{minipage} & \begin{minipage}[t]{0.08\columnwidth}\raggedright\strut
1.0.0\strut
\end{minipage} & \begin{minipage}[t]{0.49\columnwidth}\raggedright\strut
Componente para seleccionar valores numéricos.\strut
\end{minipage} & \begin{minipage}[t]{0.11\columnwidth}\raggedright\strut
Apache v2.0\strut
\end{minipage}\tabularnewline
\begin{minipage}[t]{0.18\columnwidth}\raggedright\strut
Permission Utils\strut
\end{minipage} & \begin{minipage}[t]{0.08\columnwidth}\raggedright\strut
1.0.6\strut
\end{minipage} & \begin{minipage}[t]{0.49\columnwidth}\raggedright\strut
Biblioteca que facilita la gestión de permisos en tiempo de
ejecución.\strut
\end{minipage} & \begin{minipage}[t]{0.11\columnwidth}\raggedright\strut
MIT\strut
\end{minipage}\tabularnewline
\begin{minipage}[t]{0.18\columnwidth}\raggedright\strut
JUnit\strut
\end{minipage} & \begin{minipage}[t]{0.08\columnwidth}\raggedright\strut
4.12\strut
\end{minipage} & \begin{minipage}[t]{0.49\columnwidth}\raggedright\strut
Framework para \emph{testing} unitario en Java.\strut
\end{minipage} & \begin{minipage}[t]{0.11\columnwidth}\raggedright\strut
EPL\strut
\end{minipage}\tabularnewline
\begin{minipage}[t]{0.18\columnwidth}\raggedright\strut
Mockito\strut
\end{minipage} & \begin{minipage}[t]{0.08\columnwidth}\raggedright\strut
2.0.2\strut
\end{minipage} & \begin{minipage}[t]{0.49\columnwidth}\raggedright\strut
Framework para \emph{mocking} en Java.\strut
\end{minipage} & \begin{minipage}[t]{0.11\columnwidth}\raggedright\strut
MIT\strut
\end{minipage}\tabularnewline
\begin{minipage}[t]{0.18\columnwidth}\raggedright\strut
SLF4J\strut
\end{minipage} & \begin{minipage}[t]{0.08\columnwidth}\raggedright\strut
1.7.21\strut
\end{minipage} & \begin{minipage}[t]{0.49\columnwidth}\raggedright\strut
API para \emph{logging} en Java.\strut
\end{minipage} & \begin{minipage}[t]{0.11\columnwidth}\raggedright\strut
MIT\strut
\end{minipage}\tabularnewline
\begin{minipage}[t]{0.18\columnwidth}\raggedright\strut
Apache Log4j\strut
\end{minipage} & \begin{minipage}[t]{0.08\columnwidth}\raggedright\strut
1.7.21\strut
\end{minipage} & \begin{minipage}[t]{0.49\columnwidth}\raggedright\strut
Biblioteca para \emph{logging} en Java.\strut
\end{minipage} & \begin{minipage}[t]{0.11\columnwidth}\raggedright\strut
Apache v2.0\strut
\end{minipage}\tabularnewline
\begin{minipage}[t]{0.18\columnwidth}\raggedright\strut
Android JSON\strut
\end{minipage} & \begin{minipage}[t]{0.08\columnwidth}\raggedright\strut
20160810\strut
\end{minipage} & \begin{minipage}[t]{0.49\columnwidth}\raggedright\strut
Biblioteca para trabajar con JSON.\strut
\end{minipage} & \begin{minipage}[t]{0.11\columnwidth}\raggedright\strut
Apache v2.0\strut
\end{minipage}\tabularnewline
\begin{minipage}[t]{0.18\columnwidth}\raggedright\strut
Espresso\strut
\end{minipage} & \begin{minipage}[t]{0.08\columnwidth}\raggedright\strut
2.2.2\strut
\end{minipage} & \begin{minipage}[t]{0.49\columnwidth}\raggedright\strut
Framework de \emph{testing} para Android.\strut
\end{minipage} & \begin{minipage}[t]{0.11\columnwidth}\raggedright\strut
Apache v2.0\strut
\end{minipage}\tabularnewline
\bottomrule
\caption{Dependencias del proyecto.}
\end{longtable}

Por lo tanto, tenemos que escoger una licencia para nuestro proyecto que
sea compatible con Apache v2.0, BSD, MIT y EPL. En el siguiente gráfico
mostramos la compatibilidad entre estas licencias, así como su grado de
permisividad.

\imagen{licenses_compatibility}{Compatibilidad entre licencias.}

Podemos observar que la licencia más restrictiva (en el sentido de
obligaciones a cumplir) es la \emph{Eclipse Public License} que posee la
librería JUnit.

La forma de monetización del proyecto se realizará mediante
suscripciones a una plataforma \emph{cloud} que permitirá la
sincronización entre varios dispositivos, entre otras funcionalidades.
Por lo tanto, la liberación del código del proyecto no pone en peligro
su monetización, sino todo lo contrario, abre la puerta a que la
comunidad \emph{Open Source} aporte valor adicional a nuestro proyecto.
El permitir la distribución de la app libremente y de forma gratuita
también nos es beneficioso, ya que aumenta las posibilidades de recibir
nuevas suscripciones de usuarios. Y por último, no nos importaría que
otras empresas se basaran en nuestro código fuente para desarrollar sus
productos, siempre los liberaran bajo una licencia de código abierto
para que nosotros también pudiéramos aprovechar las mejoras que hubieran
realizado.

Teniendo en cuenta todo lo anterior, la licencia que más se ajusta a
nuestras pretensiones es la \emph{GNU General Public License v3.0}, que,
de forma resumida, establece lo siguiente: \citep{license:gplv3}

\begin{longtable}[]{@{}lll@{}}
\toprule
\begin{minipage}[b]{0.19\columnwidth}\raggedright\strut
Derechos\strut
\end{minipage} & \begin{minipage}[b]{0.40\columnwidth}\raggedright\strut
Condiciones\strut
\end{minipage} & \begin{minipage}[b]{0.32\columnwidth}\raggedright\strut
Limitaciones\strut
\end{minipage}\tabularnewline
\midrule
\endhead
\begin{minipage}[t]{0.19\columnwidth}\raggedright\strut
Uso comercial.\strut
\end{minipage} & \begin{minipage}[t]{0.40\columnwidth}\raggedright\strut
Liberar código fuente.\strut
\end{minipage} & \begin{minipage}[t]{0.32\columnwidth}\raggedright\strut
Limitación de responsabilidad.\strut
\end{minipage}\tabularnewline
\begin{minipage}[t]{0.19\columnwidth}\raggedright\strut
Distribución.\strut
\end{minipage} & \begin{minipage}[t]{0.40\columnwidth}\raggedright\strut
Nota sobre la licencia y copyright.\strut
\end{minipage} & \begin{minipage}[t]{0.32\columnwidth}\raggedright\strut
Sin garantías.\strut
\end{minipage}\tabularnewline
\begin{minipage}[t]{0.19\columnwidth}\raggedright\strut
Modificación.\strut
\end{minipage} & \begin{minipage}[t]{0.40\columnwidth}\raggedright\strut
Modificaciones bajo la misma licencia.\strut
\end{minipage} & \begin{minipage}[t]{0.32\columnwidth}\raggedright\strut
\strut
\end{minipage}\tabularnewline
\begin{minipage}[t]{0.19\columnwidth}\raggedright\strut
Uso de patentes.\strut
\end{minipage} & \begin{minipage}[t]{0.40\columnwidth}\raggedright\strut
Indicar modificaciones realizadas.\strut
\end{minipage} & \begin{minipage}[t]{0.32\columnwidth}\raggedright\strut
\strut
\end{minipage}\tabularnewline
\begin{minipage}[t]{0.19\columnwidth}\raggedright\strut
Uso privado.\strut
\end{minipage} & \begin{minipage}[t]{0.40\columnwidth}\raggedright\strut
\strut
\end{minipage} & \begin{minipage}[t]{0.32\columnwidth}\raggedright\strut
\strut
\end{minipage}\tabularnewline
\bottomrule
\caption{Resumen de la licencia GLPv3.}
\end{longtable}

Sin embargo, GPL v3.0 no es compatible con la licencia EPL que posee
JUnit. Ya que, la EPL requiere que ``cualquier distribución del trabajo
conceda a todos los destinatarios una licencia para las patentes que
pudieran tener que cubrir las modificaciones que han hecho''
\citep{license:epl}. Esto supone que los destinatarios pueden añadir una
restricción adicional, hecho que prohíbe rotundamente GPL: ``{[}que el
distribuidor{]} no imponga ninguna restricción más sobre el ejercicio de
los derechos concedidos a los beneficiarios'' \citep{license:gplv3}.

Tras analizar otras licencias alternativas, no se ha encontrado ninguna
compatible con EPL y, a la vez, con nuestras pretensiones. Por lo que
finalmente se ha tomado la decisión de utilizar dos licencias para el
código fuente del proyecto. Por un lado, todo el código fuente de la
aplicación se ha licenciado bajo GPL v3.0. Mientras que el código fuente
de testeo, que hace uso de código licenciado bajo EPL (JUnit), se ha
liberado bajo licencia Apache v2.0, la cual sí que es compatible con
EPL.

\subsubsection{Documentación}\label{documentaciuxf3n}

Aunque se puede utilizar también la licencia GPL v3.0 para licenciar la
documentación, no es lo más recomendable. Ya que contiene numerosas
cláusulas que solo tienen sentido cuando se habla de código fuente. Por
ejemplo, si alguien quisiese distribuir una copia de la documentación de
forma impresa, estaría obligado a proporcionar también una copia del
código fuente.

Por lo que se ha decido utilizar una licencia \emph{Creative Commons},
las cuales están más enfocadas a licenciar este tipo de material. En
concreto, se ha elegido la \emph{Creative Commons Attribution 4.0
International} (CC-BY-4.0). Que establece lo siguiente:
\citep{license:ccby4}

\begin{longtable}[]{@{}lll@{}}
\toprule
\begin{minipage}[b]{0.17\columnwidth}\raggedright\strut
Derechos\strut
\end{minipage} & \begin{minipage}[b]{0.32\columnwidth}\raggedright\strut
Condiciones\strut
\end{minipage} & \begin{minipage}[b]{0.43\columnwidth}\raggedright\strut
Limitaciones\strut
\end{minipage}\tabularnewline
\midrule
\endhead
\begin{minipage}[t]{0.17\columnwidth}\raggedright\strut
Uso comercial.\strut
\end{minipage} & \begin{minipage}[t]{0.32\columnwidth}\raggedright\strut
Nota sobre la licencia y copyright.\strut
\end{minipage} & \begin{minipage}[t]{0.43\columnwidth}\raggedright\strut
Limitación de responsabilidad.\strut
\end{minipage}\tabularnewline
\begin{minipage}[t]{0.17\columnwidth}\raggedright\strut
Distribución.\strut
\end{minipage} & \begin{minipage}[t]{0.32\columnwidth}\raggedright\strut
Indicar modificaciones realizadas.\strut
\end{minipage} & \begin{minipage}[t]{0.43\columnwidth}\raggedright\strut
Sin garantías.\strut
\end{minipage}\tabularnewline
\begin{minipage}[t]{0.17\columnwidth}\raggedright\strut
Modificación.\strut
\end{minipage} & \begin{minipage}[t]{0.32\columnwidth}\raggedright\strut
\strut
\end{minipage} & \begin{minipage}[t]{0.43\columnwidth}\raggedright\strut
No proporciona derechos sobre marcas registradas.\strut
\end{minipage}\tabularnewline
\begin{minipage}[t]{0.17\columnwidth}\raggedright\strut
Uso privado.\strut
\end{minipage} & \begin{minipage}[t]{0.32\columnwidth}\raggedright\strut
\strut
\end{minipage} & \begin{minipage}[t]{0.43\columnwidth}\raggedright\strut
No proporciona derechos sobre patentes.\strut
\end{minipage}\tabularnewline
\bottomrule
\caption{Resumen de la licencia CC-BY-4.0.}
\end{longtable}

\subsubsection{Imágenes y vídeos}\label{imuxe1genes-y-vuxeddeos}

En la documentación no se ha utilizado ninguna imagen de terceros, todas
las imágenes son propias del proyecto y cuentan con la misma licencia
que la documentación (CC-BY-4.0).

El \emph{dataset} de vídeos de prueba también se encuentra bajo la misma
licencia.

Por otro lado, en la aplicación se han utilizado dos fuentes de imágenes
de terceros:

\begin{longtable}[]{@{}lll@{}}
\toprule
\begin{minipage}[b]{0.28\columnwidth}\raggedright\strut
Fuente\strut
\end{minipage} & \begin{minipage}[b]{0.46\columnwidth}\raggedright\strut
Descripción\strut
\end{minipage} & \begin{minipage}[b]{0.17\columnwidth}\raggedright\strut
Licencia\strut
\end{minipage}\tabularnewline
\midrule
\endhead
\begin{minipage}[t]{0.28\columnwidth}\raggedright\strut
Material design icons\strut
\end{minipage} & \begin{minipage}[t]{0.46\columnwidth}\raggedright\strut
Conjunto de iconos oficial de Google.\strut
\end{minipage} & \begin{minipage}[t]{0.17\columnwidth}\raggedright\strut
Apache v2.0\strut
\end{minipage}\tabularnewline
\begin{minipage}[t]{0.28\columnwidth}\raggedright\strut
Simple Weather Icons\strut
\end{minipage} & \begin{minipage}[t]{0.46\columnwidth}\raggedright\strut
Conjunto de iconos meteorológicos.\strut
\end{minipage} & \begin{minipage}[t]{0.17\columnwidth}\raggedright\strut
Apache v2.0\strut
\end{minipage}\tabularnewline
\bottomrule
\caption{Fuentes de imágenes de terceros.}
\end{longtable}

Aunque ambos autores renuncian a la obligación de especificar
explícitamente su autoría, se les ha mencionado en la sección
``Licencias de software libre'' de la aplicación.

El resto de imágenes y gráficos utilizados son de autoría propia y se
distribuyen también bajo CC-BY-4.0.3.

\subsubsection{Resumen}\label{resumen-1}

En la siguiente tabla se resumen las licencias que posee el proyecto.

\begin{longtable}[]{@{}ll@{}}
\toprule
\begin{minipage}[b]{0.31\columnwidth}\raggedright\strut
Recurso\strut
\end{minipage} & \begin{minipage}[b]{0.21\columnwidth}\raggedright\strut
Licencia\strut
\end{minipage}\tabularnewline
\midrule
\endhead
\begin{minipage}[t]{0.31\columnwidth}\raggedright\strut
Código fuente app\strut
\end{minipage} & \begin{minipage}[t]{0.21\columnwidth}\raggedright\strut
GPLv3\strut
\end{minipage}\tabularnewline
\begin{minipage}[t]{0.31\columnwidth}\raggedright\strut
Código fuente tests\strut
\end{minipage} & \begin{minipage}[t]{0.21\columnwidth}\raggedright\strut
Apache v2.0\strut
\end{minipage}\tabularnewline
\begin{minipage}[t]{0.31\columnwidth}\raggedright\strut
Documentación\strut
\end{minipage} & \begin{minipage}[t]{0.21\columnwidth}\raggedright\strut
CC-BY-4.0\strut
\end{minipage}\tabularnewline
\begin{minipage}[t]{0.31\columnwidth}\raggedright\strut
Imágenes\strut
\end{minipage} & \begin{minipage}[t]{0.21\columnwidth}\raggedright\strut
CC-BY-4.0\strut
\end{minipage}\tabularnewline
\begin{minipage}[t]{0.31\columnwidth}\raggedright\strut
Vídeos\strut
\end{minipage} & \begin{minipage}[t]{0.21\columnwidth}\raggedright\strut
CC-BY-4.0\strut
\end{minipage}\tabularnewline
\bottomrule
\caption{Resumen de las licencias del proyecto.}
\end{longtable}
