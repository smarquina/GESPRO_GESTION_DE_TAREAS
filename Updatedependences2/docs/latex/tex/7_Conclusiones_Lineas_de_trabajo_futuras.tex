\capitulo{7}{Conclusiones y Líneas de trabajo futuras}

En esta sección se exponen las conclusiones derivadas del trabajo, así
como las posibles líneas de trabajo futuras por las que se puede dar
continuidad al proyecto.

\section{Conclusiones}\label{conclusiones}

Tras el desarrollo del proyecto podemos extraer las siguientes
conclusiones:

\begin{itemize}
\tightlist
\item
  El objetivo general del proyecto se ha cumplido satisfactoriamente.
  Ahora los apicultores cuentan con una aplicación Android que no solo
  les permite gestionar sus colmenares, sino que además les brinda la
  posibilidad de monitorizar la actividad de vuelo de sus colmenas e
  interpretar los datos recogidos.
\item
  El haber utilizado el ecosistema Android para la realización del
  proyecto ha aportado ciertas ventajas tanto en herramientas de
  desarrollo, como en la distribución de la aplicación. Sin embargo,
  también ha supuesto un desafío al tener que desarrollar un algoritmo
  de una cierta complejidad para dispositivos con recursos bastante
  limitados.
\item
  El proyecto ha abarcado gran parte de los conocimientos adquiridos
  durante el grado. Además, ha requerido el aprendizaje de muchos otros
  como la visión artificial, OpenCV, Android, etc.
\item
  Durante el proyecto se han utilizado un gran número de tecnologías y
  herramientas. La mayoría de ellas han contribuido a mejorar la calidad
  del producto final o de los procesos intermedios. No obstante, algunas
  de ellas han supuesto una sobrecarga importante, como sucedió en el
  caso de la documentación. Aun así, el conocimiento adquirido de todas
  ellas será de mucha utilidad en proyectos futuros.
\item
  Gracias a la parte de investigación que posee el proyecto, se ha
  aprendido a realizar búsquedas bibliográficas y a familiarizarse con
  la lectura de artículos científicos.
\item
  La utilización de varios servicios de integración continua nos ha
  permitido la detección temprana de defectos en el \emph{software},
  reduciendo el impacto de estos y dando lugar a un código de mayor
  calidad.
\item
  Es muy difícil estimar la duración de tareas de investigación o tareas
  sobre las que no se posee un conocimiento previo. Sin embargo, el
  haber aplicado una metodología ágil nos ha permitido ser más flexibles
  ante los cambios. Y finalmente, se ha completado satisfactoriamente el
  proyecto en el plazo establecido.
\end{itemize}

\section{Líneas de trabajo futuras}\label{luxedneas-de-trabajo-futuras}

En primer lugar, comentar que la entrega del Trabajo de Fin de Grado
solo es un hito en el camino del proyecto, ya que su desarrollo
prosigue. A continuación, se resume el \emph{roadmap} del proyecto:

\begin{itemize}
\tightlist
\item  En la columna ``\emph{New issues}'' del gestor de tareas se encuentran
  las nuevas funcionalidades en las que se trabajarán en los próximos
  meses. Entre ellas se encuentran: dar soporte a operaciones por lotes
  orientadas a apicultores profesionales con un gran número de colmenas,
  permitir la exportación de los datos almacenados, añadir informes de
  revisión de colmenas, varias mejoras de usabilidad y diseño, etc.
\item
  Por otro lado, para la beca de colaboración con departamentos se
  trabajará en la mejora del algoritmo de monitorización actual. Se
  probarán nuevas técnicas de detección y \emph{tracking}, y se
  estudiará su viabilidad para ser ejecutadas en dispositivos móviles.
\item
  Para finales de año se planea tener desarrollada una plataforma en la
  nube que sincronice los datos de varios dispositivos y permita el
  acceso a estos mediante una aplicación web. Se planea monetizar el
  proyecto mediante la subscripción a esta plataforma.
\item
  Se considerará la opción de migrar la aplicación a otras plataformas.
\item
  En el futuro, se podrían explotar los datos almacenados en la
  plataforma para intentar desarrollar algoritmos que predigan ciertos
  eventos (enfermedades, enjambrazón, etc.) y que permitan al apicultor
  anticiparse a estos.
\end{itemize}
