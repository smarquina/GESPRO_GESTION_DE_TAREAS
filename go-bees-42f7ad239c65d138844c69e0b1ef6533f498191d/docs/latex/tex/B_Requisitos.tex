\apendice{Especificación de Requisitos}

\section{Introducción}\label{introduccion-requisitos}

Este anexo recoge la especificación de requisitos que define el
comportamiento del sistema desarrollado. Posee un doble objetivo: servir
como documento contractual entre el cliente y el equipo de desarrollo y
como documentación correspondiente al análisis a la aplicación.

Se han seguido las recomendaciones del estándar IEEE
830-1998, que manifiesta que una buena especificación de requisitos
\emph{software} debe ser: \citep{ieee_830_1998}

\begin{itemize}
\tightlist
\item
  \textbf{Completa}: todos los requerimientos deben estar reflejados en
  ella y todas las referencias deben estar definidas.
\item
  \textbf{Consistente}: debe ser coherente con los propios
  requerimientos y también con otros documentos de especificación.
\item
  \textbf{Inequívoca}: la redacción debe ser clara de modo que no se
  pueda mal interpretar.
\item
  \textbf{Correcta}: el software debe cumplir con los requisitos de la
  especificación.
\item
  \textbf{Trazable}\emph{: s}e refiere a la posibilidad de verificar la
  historia, ubicación o aplicación de un ítem a través de su
  identificación almacenada y documentada.
\item
  \textbf{Priorizable}: los requisitos deben poder organizarse
  jerárquicamente según su relevancia para el negocio y clasificándolos
  en esenciales, condicionales y opcionales.
\item
  \textbf{Modificable}: aunque todo requerimiento es modificable, se
  refiere a que debe ser fácilmente modificable.
\item
  \textbf{Verificable}: debe existir un método finito sin costo para
  poder probarlo.
\end{itemize}

\section{Objetivos generales}\label{objetivos-generales}

El proyecto persigue los siguientes objetivos generales:

\begin{itemize}
\tightlist
\item
  Desarrollar una aplicación para \emph{smartphones} que permita la
  monitorización de la actividad de vuelo de una colmena a través de su
  cámara.
\item
  Facilitar la interpretación de los datos recogidos mediante
  representaciones gráficas.
\item
  Aportar información extra a los datos de actividad que ayude en la
  toma de decisiones.
\item
  Almacenar todos los datos generados de forma estructurada y fácilmente
  accesible.
\end{itemize}

\section{Catálogo de requisitos}\label{catalogo-de-requisitos}

A continuación, se enumeran los requisitos específicos derivados de los
objetivos generales del proyecto.

\subsection{Requisitos funcionales}\label{requisitos-funcionales}

\begin{itemize}
\tightlist
\item
  \textbf{RF-1 Gestión de colmenares:} la aplicación tiene que ser capaz
  de gestionar colmenares.

  \begin{itemize}
  \tightlist
  \item
    \textbf{RF-1.1 Añadir colmenar:} el usuario debe poder añadir un
    nuevo colmenar con un nombre, una localización y unas notas
    específicas.

    \begin{itemize}
    \tightlist
    \item
      \textbf{RF-1.1.1: Obtener localización:} la aplicación tiene que
      ser capaz de obtener la localización actual del usuario.
    \end{itemize}
  \item
    \textbf{RF-1.2 Editar colmenar:} el usuario debe poder editar la
    información de un colmenar ya existente.

    \begin{itemize}
    \tightlist
    \item
      \textbf{RF-1.2.1: Obtener localización:} la aplicación tiene que
      ser capaz de obtener la localización actual del usuario.
    \end{itemize}
  \item
    \textbf{RF-1.3 Eliminar colmenar:} el usuario debe poder eliminar un
    colmenar ya existente junto con toda su información asociada.
  \item
    \textbf{RF-1.4 Listar colmenares:} el usuario debe poder listar
    todos los colmenares existentes.

    \begin{itemize}
    \tightlist
    \item
      \textbf{RF-1.4.1 Obtención de información meteorológica:} la
      aplicación tiene que ser capaz de obtener la información
      meteorológica de cada uno de los colmenares.
    \end{itemize}
  \item
    \textbf{RF-1.5 Ver colmenar:} el usuario debe poder visualizar toda
    la información relativa a un determinado colmenar.

    \begin{itemize}
    \tightlist
    \item
      \textbf{RF-1.5.1 Obtención de información meteorológica:} la
      aplicación tiene que ser capaz de obtener la información
      meteorológica relativa a un determinado colmenar.
    \end{itemize}
  \end{itemize}
\item
  \textbf{RF-2 Gestión de colmenas:} la aplicación tiene que ser capaz
  de gestionar colmenas.

  \begin{itemize}
  \tightlist
  \item
    \textbf{RF-2.1 Añadir colmena:} el usuario debe poder añadir una
    nueva colmena con un nombre y unas notas específicas.
  \item
    \textbf{RF-2.2 Editar colmena:} el usuario debe poder editar la
    información de una colmena ya existente.
  \item
    \textbf{RF-2.3 Eliminar colmena:} el usuario debe poder eliminar una
    colmena ya existente junto con toda su información asociada.
  \item
    \textbf{RF-2.4 Listar colmenas:} el usuario debe poder listar todas
    las colmenas existentes en un determinado colmenar.
  \item
    \textbf{RF-2.5 Ver colmena:} el usuario debe poder visualizar toda
    la información relativa a una determinada colmena.
  \end{itemize}
\item
  \textbf{RF-3 Gestión de grabaciones:} la aplicación tiene que ser
  capaz de gestionar grabaciones.

  \begin{itemize}
  \tightlist
  \item
    \textbf{RF-3.1 Añadir grabación:} la aplicación tiene que ser capaz
    de crear una nueva grabación a partir de los datos de
    monitorización.
  \item
    \textbf{RF-3.2 Eliminar grabación:} el usuario debe poder eliminar
    una grabación ya existente junto con toda su información asociada.
  \item
    \textbf{RF-3.3 Listar grabaciones:} el usuario debe poder listar
    todas las grabaciones existentes de una determinada colmena.
  \item
    \textbf{RF-3.4 Ver grabación:} el usuario debe poder visualizar toda
    la información relativa a una determinada grabación.
  \end{itemize}
\item
  \textbf{RF-4 Monitorización de la actividad de vuelo:} el usuario
  tiene que ser capaz de monitorizar la actividad de vuelo de una
  colmena a partir de una determinada parametrización de esta.

  \begin{itemize}
  \tightlist
  \item
    \textbf{RF-4.1 Previsualización:} el usuario debe poder
    previsualizar la salida del algoritmo de conteo de abejas.
  \item
    \textbf{RF-4.2 Configurar monitorización:} el usuario debe poder
    configurar todos los parámetros relativos a la monitorización.
  \item
    \textbf{RF-4.3 Obtención de información meteorológica:} la
    aplicación tiene que ser capaz de obtener la información
    meteorológica relativa a un determinado colmenar.
  \end{itemize}
\item
  \textbf{RF-5 Configuración de la aplicación:} el usuario debe poder
  configurar todos los parámetros disponibles en la aplicación, como el
  idioma o las unidades meteorológicas.
\item
  \textbf{RF-6 Ayuda de la aplicación:} el usuario debe poder obtener
  ayuda sobre cada una de las funcionalidades de la aplicación.
\item
  \textbf{RF-7 Información de la aplicación:} el usuario debe poder
  obtener información sobre la aplicación, compartirla o enviar
  sugerencias.
\end{itemize}

\subsection{Requisitos no funcionales}\label{requisitos-no-funcionales}

\begin{itemize}
\tightlist
\item
  \textbf{RNF-1 Usabilidad:} la aplicación debe ser intuitiva, con una
  curva baja de aprendizaje, errores explicativos y adaptada al entorno
  de trabajo.
\item
  \textbf{RNF-2 Rendimiento:} la aplicación tiene que tener unos tiempos
  de carga y procesado aceptables en un dispositivo móvil de gama media.
  La pantalla nunca deberá quedar congelada.
\item
  \textbf{RNF-3 Capacidad y Escalabilidad:} la aplicación tiene que
  estar preparada para una recogida de datos continuada y debe permitir
  la adición de nuevas funcionalidades de forma sencilla.
\item
  \textbf{RNF-4 Disponibilidad:} la aplicación debe estar siempre
  disponible para su uso, independientemente de la localización, la no
  disponibilidad de internet, o cualquier otro factor.
\item
  \textbf{RNF-5 Seguridad:} la aplicación debe gestionar de forma
  adecuada todos los datos de carácter sensible, como claves,
  \emph{tokens}, etc.
\item
  \textbf{RNF-6 Mantenibilidad}: la aplicación debe ser desarrollada de
  acuerdo a algún patrón arquitectónico estándar que asegure
  escalabilidad, portabilidad, testabilidad, etc. Además, tiene que
  cumplir los estándares de código de Android.
\item
  \textbf{RNF-7 Soporte}: la aplicación debe dar soporte a versiones
  mayores o iguales a Android 4.4 (\emph{KitKat}).
\item
  \textbf{RNF-8 Monitorización}: la aplicación debe monitorizar
  correctamente la actividad de vuelo de una colmena cuando el
  dispositivo se coloca en posición cenital a la colmena, sobre un
  soporte estático y con un fondo claro y uniforme.
\item
  \textbf{RNF-9 Internacionalización}: la aplicación deberá estar
  preparada para soportar varios idiomas, localizando textos, unidades
  de medida, imágenes, etc.
\end{itemize}

\section{Especificación de
requisitos}\label{especificacion-de-requisitos-1}

En esta sección se mostrará el diagrama de casos de uso resultante y se
desarrollará cada uno de ellos.

\begin{landscape}
\subsection{Diagrama de casos de uso}\label{diagrama-de-casos-de-uso}
\imagenAncho{use_cases_diagram}{Diagrama de casos de uso.}{1.35}
\end{landscape}

\subsection{Actores}\label{actores}

Solo interactuará con el sistema un actor, que se corresponderá con la
figura del apicultor.

\subsection{Casos de uso}\label{casos-de-uso}

\begin{longtable}[H]{@{}ll@{}}
\toprule
\begin{minipage}[b]{0.23\columnwidth}\raggedright\strut
\textbf{CU-01}\strut
\end{minipage} & \begin{minipage}[b]{0.71\columnwidth}\raggedright\strut
\textbf{Gestión de colmenares}\strut
\end{minipage}\tabularnewline
\midrule
\endhead
\begin{minipage}[t]{0.23\columnwidth}\raggedright\strut
\textbf{Versión}\strut
\end{minipage} & \begin{minipage}[t]{0.71\columnwidth}\raggedright\strut
1.0\strut
\end{minipage}\tabularnewline
\begin{minipage}[t]{0.23\columnwidth}\raggedright\strut
\textbf{Autor}\strut
\end{minipage} & \begin{minipage}[t]{0.71\columnwidth}\raggedright\strut
David Miguel Lozano\strut
\end{minipage}\tabularnewline
\begin{minipage}[t]{0.23\columnwidth}\raggedright\strut
\textbf{Requisitos asociados}\strut
\end{minipage} & \begin{minipage}[t]{0.71\columnwidth}\raggedright\strut
RF-1, RF-1.1, RF-1.1.1, RF-1.2, RF-1.2.1, RF-1.3, RF-1.4, RF-1.5,
RF-1.5.1\strut
\end{minipage}\tabularnewline
\begin{minipage}[t]{0.23\columnwidth}\raggedright\strut
\textbf{Descripción}\strut
\end{minipage} & \begin{minipage}[t]{0.71\columnwidth}\raggedright\strut
Permite al usuario gestionar sus colmenares.\strut
\end{minipage}\tabularnewline
\begin{minipage}[t]{0.23\columnwidth}\raggedright\strut
\textbf{Precondición}\strut
\end{minipage} & \begin{minipage}[t]{0.71\columnwidth}\raggedright\strut
La base de datos se encuentra disponible.\strut
\end{minipage}\tabularnewline
\begin{minipage}[t]{0.23\columnwidth}\raggedright\strut
\textbf{Acciones}\strut
\end{minipage} & \begin{minipage}[t]{0.71\columnwidth}\raggedright\strut
\begin{enumerate}
\def\labelenumi{\arabic{enumi}.}
\tightlist
\item
  El usuario entra en la aplicación.
\item
  Se listan todos los colmenares.
\item
  Por cada colmenar se da la opción de ver detalle, editar o eliminar.
\item
  Se muestra un botón para añadir un colmenar.
\end{enumerate}\strut
\end{minipage}\tabularnewline
\begin{minipage}[t]{0.23\columnwidth}\raggedright\strut
\textbf{Postcondición}\strut
\end{minipage} & \begin{minipage}[t]{0.71\columnwidth}\raggedright\strut
El número de colmenares listado es igual al número de colmenares en la
base de datos.\strut
\end{minipage}\tabularnewline
\begin{minipage}[t]{0.23\columnwidth}\raggedright\strut
\textbf{Excepciones}\strut
\end{minipage} & \begin{minipage}[t]{0.71\columnwidth}\raggedright\strut
\begin{itemize}
\tightlist
\item
  Error al cargar colmenares (mensaje).
\item
  No existe ningún colmenar (vista especial).
\end{itemize}\strut
\end{minipage}\tabularnewline
\begin{minipage}[t]{0.23\columnwidth}\raggedright\strut
\textbf{Importancia}\strut
\end{minipage} & \begin{minipage}[t]{0.71\columnwidth}\raggedright\strut
Alta\strut
\end{minipage}\tabularnewline
\bottomrule
\caption{CU-01 Gestión de colmenares.}
\end{longtable}

\begin{longtable}[H]{@{}ll@{}}
\toprule
\begin{minipage}[b]{0.24\columnwidth}\raggedright\strut
\textbf{CU-02}\strut
\end{minipage} & \begin{minipage}[b]{0.71\columnwidth}\raggedright\strut
\textbf{Añadir colmenar}\strut
\end{minipage}\tabularnewline
\midrule
\endhead
\begin{minipage}[t]{0.24\columnwidth}\raggedright\strut
\textbf{Versión}\strut
\end{minipage} & \begin{minipage}[t]{0.71\columnwidth}\raggedright\strut
1.0\strut
\end{minipage}\tabularnewline
\begin{minipage}[t]{0.24\columnwidth}\raggedright\strut
\textbf{Autor}\strut
\end{minipage} & \begin{minipage}[t]{0.71\columnwidth}\raggedright\strut
David Miguel Lozano\strut
\end{minipage}\tabularnewline
\begin{minipage}[t]{0.24\columnwidth}\raggedright\strut
\textbf{Requisitos asociados}\strut
\end{minipage} & \begin{minipage}[t]{0.71\columnwidth}\raggedright\strut
RF-1.1, RF-1.1.1\strut
\end{minipage}\tabularnewline
\begin{minipage}[t]{0.24\columnwidth}\raggedright\strut
\textbf{Descripción}\strut
\end{minipage} & \begin{minipage}[t]{0.71\columnwidth}\raggedright\strut
Permite al usuario añadir un nuevo colmenar.\strut
\end{minipage}\tabularnewline
\begin{minipage}[t]{0.24\columnwidth}\raggedright\strut
\textbf{Precondición}\strut
\end{minipage} & \begin{minipage}[t]{0.71\columnwidth}\raggedright\strut
La base de datos se encuentra disponible.\strut
\end{minipage}\tabularnewline
\begin{minipage}[t]{0.24\columnwidth}\raggedright\strut
\textbf{Acciones}\strut
\end{minipage} & \begin{minipage}[t]{0.71\columnwidth}\raggedright\strut
\begin{enumerate}
\def\labelenumi{\arabic{enumi}.}
\tightlist
\item
  El usuario presiona en el botón de añadir colmenar.
\item
  Se muestra el formulario para introducir los datos del colmenar.
\item
  El usuario introduce el nombre.
\item
  El usuario pulsa obtener localización (opcional).

  \begin{enumerate}
  \def\labelenumii{\alph{enumii}.}
  \tightlist
  \item
    Se obtiene la localización del usuario.
  \end{enumerate}
\item
  El usuario introduce notas sobre el colmenar (opcional).
\item
  El usuario pulsa el botón de aceptar.
\item
  Si no hay ningún error, se guarda un nuevo colmenar con los datos
  introducidos.
\item
  Volver a Gestión de colmenares.
\end{enumerate}\strut
\end{minipage}\tabularnewline
\begin{minipage}[t]{0.24\columnwidth}\raggedright\strut
\textbf{Postcondición}\strut
\end{minipage} & \begin{minipage}[t]{0.71\columnwidth}\raggedright\strut
Existe un colmenar más en la base de datos.\strut
\end{minipage}\tabularnewline
\begin{minipage}[t]{0.24\columnwidth}\raggedright\strut
\textbf{Excepciones}\strut
\end{minipage} & \begin{minipage}[t]{0.71\columnwidth}\raggedright\strut
\begin{itemize}
\tightlist
\item
  Error al guardar colmenar (mensaje).
\item
  No se ha introducido nombre del colmenar (resaltar).
\end{itemize}\strut
\end{minipage}\tabularnewline
\begin{minipage}[t]{0.24\columnwidth}\raggedright\strut
\textbf{Importancia}\strut
\end{minipage} & \begin{minipage}[t]{0.71\columnwidth}\raggedright\strut
Alta\strut
\end{minipage}\tabularnewline
\bottomrule
\caption{CU-02 Añadir colmenar.}
\end{longtable}

\begin{longtable}[H]{@{}ll@{}}
\toprule
\begin{minipage}[b]{0.26\columnwidth}\raggedright\strut
\textbf{CU-03}\strut
\end{minipage} & \begin{minipage}[b]{0.68\columnwidth}\raggedright\strut
\textbf{Editar colmenar}\strut
\end{minipage}\tabularnewline
\midrule
\endhead
\begin{minipage}[t]{0.26\columnwidth}\raggedright\strut
\textbf{Versión}\strut
\end{minipage} & \begin{minipage}[t]{0.68\columnwidth}\raggedright\strut
1.0\strut
\end{minipage}\tabularnewline
\begin{minipage}[t]{0.26\columnwidth}\raggedright\strut
\textbf{Autor}\strut
\end{minipage} & \begin{minipage}[t]{0.68\columnwidth}\raggedright\strut
David Miguel Lozano\strut
\end{minipage}\tabularnewline
\begin{minipage}[t]{0.26\columnwidth}\raggedright\strut
\textbf{Requisitos asociados}\strut
\end{minipage} & \begin{minipage}[t]{0.68\columnwidth}\raggedright\strut
RF-1.2, RF-1.2.1\strut
\end{minipage}\tabularnewline
\begin{minipage}[t]{0.26\columnwidth}\raggedright\strut
\textbf{Descripción}\strut
\end{minipage} & \begin{minipage}[t]{0.68\columnwidth}\raggedright\strut
Permite al usuario editar un colmenar ya existente.\strut
\end{minipage}\tabularnewline
\begin{minipage}[t]{0.26\columnwidth}\raggedright\strut
\textbf{Precondición}\strut
\end{minipage} & \begin{minipage}[t]{0.68\columnwidth}\raggedright\strut
La base de datos se encuentra disponible.

El colmenar a editar existe.\strut
\end{minipage}\tabularnewline
\begin{minipage}[t]{0.26\columnwidth}\raggedright\strut
\textbf{Acciones}\strut
\end{minipage} & \begin{minipage}[t]{0.68\columnwidth}\raggedright\strut
\begin{enumerate}
\def\labelenumi{\arabic{enumi}.}
\tightlist
\item
  El usuario selecciona un colmenar para editar.
\item
  Se obtienen los datos del colmenar de la base de datos.
\item
  Se rellena el formulario de edición con los datos del colmenar.
\item
  El usuario edita alguno de los campos.
\item
  Si el usuario pulsa obtener localización.

  \begin{enumerate}
  \def\labelenumii{\alph{enumii}.}
  \tightlist
  \item
    Se obtiene la localización del usuario.
  \end{enumerate}
\item
  El usuario pulsa el botón aceptar.
\item
  Si no hay ningún error, se actualiza el colmenar en la base de datos.
\end{enumerate}\strut
\end{minipage}\tabularnewline
\begin{minipage}[t]{0.26\columnwidth}\raggedright\strut
\textbf{Postcondición}\strut
\end{minipage} & \begin{minipage}[t]{0.68\columnwidth}\raggedright\strut
La información del colmenar en la base de datos ha sido
actualizada.\strut
\end{minipage}\tabularnewline
\begin{minipage}[t]{0.26\columnwidth}\raggedright\strut
\textbf{Excepciones}\strut
\end{minipage} & \begin{minipage}[t]{0.68\columnwidth}\raggedright\strut
\begin{itemize}
\tightlist
\item
  Error al guardar colmenar (mensaje).
\item
  No se ha introducido nombre del colmenar (resaltar).
\end{itemize}\strut
\end{minipage}\tabularnewline
\begin{minipage}[t]{0.26\columnwidth}\raggedright\strut
\textbf{Importancia}\strut
\end{minipage} & \begin{minipage}[t]{0.68\columnwidth}\raggedright\strut
Alta\strut
\end{minipage}\tabularnewline
\bottomrule
\caption{CU-03 Editar colmenar.}
\end{longtable}

\begin{longtable}[H]{@{}ll@{}}
\toprule
\begin{minipage}[b]{0.29\columnwidth}\raggedright\strut
\textbf{CU-04}\strut
\end{minipage} & \begin{minipage}[b]{0.65\columnwidth}\raggedright\strut
\textbf{Eliminar colmenar}\strut
\end{minipage}\tabularnewline
\midrule
\endhead
\begin{minipage}[t]{0.29\columnwidth}\raggedright\strut
\textbf{Versión}\strut
\end{minipage} & \begin{minipage}[t]{0.65\columnwidth}\raggedright\strut
1.0\strut
\end{minipage}\tabularnewline
\begin{minipage}[t]{0.29\columnwidth}\raggedright\strut
\textbf{Autor}\strut
\end{minipage} & \begin{minipage}[t]{0.65\columnwidth}\raggedright\strut
David Miguel Lozano\strut
\end{minipage}\tabularnewline
\begin{minipage}[t]{0.29\columnwidth}\raggedright\strut
\textbf{Requisitos asociados}\strut
\end{minipage} & \begin{minipage}[t]{0.65\columnwidth}\raggedright\strut
RF-1.3\strut
\end{minipage}\tabularnewline
\begin{minipage}[t]{0.29\columnwidth}\raggedright\strut
\textbf{Descripción}\strut
\end{minipage} & \begin{minipage}[t]{0.65\columnwidth}\raggedright\strut
Permite al usuario eliminar un colmenar ya existente.\strut
\end{minipage}\tabularnewline
\begin{minipage}[t]{0.29\columnwidth}\raggedright\strut
\textbf{Precondición}\strut
\end{minipage} & \begin{minipage}[t]{0.65\columnwidth}\raggedright\strut
La base de datos se encuentra disponible.

El colmenar a eliminar existe.\strut
\end{minipage}\tabularnewline
\begin{minipage}[t]{0.29\columnwidth}\raggedright\strut
\textbf{Acciones}\strut
\end{minipage} & \begin{minipage}[t]{0.65\columnwidth}\raggedright\strut
\begin{enumerate}
\def\labelenumi{\arabic{enumi}.}
\tightlist
\item
  El usuario selecciona un colmenar para eliminar.
\item
  Se eliminan los datos de ese colmenar de la base de datos.
\item
  Se elimina el colmenar de la vista.
\item
  Se informa al usuario.
\end{enumerate}\strut
\end{minipage}\tabularnewline
\begin{minipage}[t]{0.29\columnwidth}\raggedright\strut
\textbf{Postcondición}\strut
\end{minipage} & \begin{minipage}[t]{0.65\columnwidth}\raggedright\strut
Existe un colmenar menos en la base de datos.\strut
\end{minipage}\tabularnewline
\begin{minipage}[t]{0.29\columnwidth}\raggedright\strut
\textbf{Excepciones}\strut
\end{minipage} & \begin{minipage}[t]{0.65\columnwidth}\raggedright\strut
\begin{itemize}
\tightlist
\item
  Error al eliminar colmenar (mensaje).
\end{itemize}\strut
\end{minipage}\tabularnewline
\begin{minipage}[t]{0.29\columnwidth}\raggedright\strut
\textbf{Importancia}\strut
\end{minipage} & \begin{minipage}[t]{0.65\columnwidth}\raggedright\strut
Alta\strut
\end{minipage}\tabularnewline
\bottomrule
\caption{CU-04 Eliminar colmenar.}
\end{longtable}

\begin{longtable}[H]{@{}ll@{}}
\toprule
\begin{minipage}[b]{0.26\columnwidth}\raggedright\strut
\textbf{CU-05}\strut
\end{minipage} & \begin{minipage}[b]{0.68\columnwidth}\raggedright\strut
\textbf{Listar colmenares}\strut
\end{minipage}\tabularnewline
\midrule
\endhead
\begin{minipage}[t]{0.26\columnwidth}\raggedright\strut
\textbf{Versión}\strut
\end{minipage} & \begin{minipage}[t]{0.68\columnwidth}\raggedright\strut
1.0\strut
\end{minipage}\tabularnewline
\begin{minipage}[t]{0.26\columnwidth}\raggedright\strut
\textbf{Autor}\strut
\end{minipage} & \begin{minipage}[t]{0.68\columnwidth}\raggedright\strut
David Miguel Lozano\strut
\end{minipage}\tabularnewline
\begin{minipage}[t]{0.26\columnwidth}\raggedright\strut
\textbf{Requisitos asociados}\strut
\end{minipage} & \begin{minipage}[t]{0.68\columnwidth}\raggedright\strut
RF-1.4, RF-1.4.1\strut
\end{minipage}\tabularnewline
\begin{minipage}[t]{0.26\columnwidth}\raggedright\strut
\textbf{Descripción}\strut
\end{minipage} & \begin{minipage}[t]{0.68\columnwidth}\raggedright\strut
Permite al usuario listar todos sus colmenares. Por cada colmenar se
muestra el nombre, el número de colmenas y la condición meteorológica y
temperatura actuales.\strut
\end{minipage}\tabularnewline
\begin{minipage}[t]{0.26\columnwidth}\raggedright\strut
\textbf{Precondición}\strut
\end{minipage} & \begin{minipage}[t]{0.68\columnwidth}\raggedright\strut
La base de datos se encuentra disponible.\strut
\end{minipage}\tabularnewline
\begin{minipage}[t]{0.26\columnwidth}\raggedright\strut
\textbf{Acciones}\strut
\end{minipage} & \begin{minipage}[t]{0.68\columnwidth}\raggedright\strut
\begin{enumerate}
\def\labelenumi{\arabic{enumi}.}
\tightlist
\item
  El usuario accede a Gestionar Colmenares.
\item
  Se obtienen todos los colmenares de la base de datos.
\item
  Se actualiza su información meteorológica si no se dispone de esta o
  la que se dispone es de hace más de 15 minutos.
\item
  Se muestran la lista de colmenares. Cada elemento de la lista posee el
  nombre del colmenar, el número de colmenas y la condición
  meteorológica y temperatura de ese colmenar.
\end{enumerate}\strut
\end{minipage}\tabularnewline
\begin{minipage}[t]{0.26\columnwidth}\raggedright\strut
\textbf{Postcondición}\strut
\end{minipage} & \begin{minipage}[t]{0.68\columnwidth}\raggedright\strut
-\strut
\end{minipage}\tabularnewline
\begin{minipage}[t]{0.26\columnwidth}\raggedright\strut
\textbf{Excepciones}\strut
\end{minipage} & \begin{minipage}[t]{0.68\columnwidth}\raggedright\strut
\begin{itemize}
\tightlist
\item
  Error al cargar colmenares (mensaje).
\item
  No existen colmenares (vista especial).
\item
  No existe conexión a internet (mensaje).
\item
  Error al recuperar la información meteorológica (mensaje).
\end{itemize}\strut
\end{minipage}\tabularnewline
\begin{minipage}[t]{0.26\columnwidth}\raggedright\strut
\textbf{Importancia}\strut
\end{minipage} & \begin{minipage}[t]{0.68\columnwidth}\raggedright\strut
Alta\strut
\end{minipage}\tabularnewline
\bottomrule
\caption{CU-05 Listar colmenares.}
\end{longtable}

\begin{longtable}[H]{@{}ll@{}}
\toprule
\begin{minipage}[b]{0.268\columnwidth}\raggedright\strut
\textbf{CU-06}\strut
\end{minipage} & \begin{minipage}[b]{0.76\columnwidth}\raggedright\strut
\textbf{Ver colmenar}\strut
\end{minipage}\tabularnewline
\midrule
\endhead
\begin{minipage}[t]{0.268\columnwidth}\raggedright\strut
\textbf{Versión}\strut
\end{minipage} & \begin{minipage}[t]{0.76\columnwidth}\raggedright\strut
1.0\strut
\end{minipage}\tabularnewline
\begin{minipage}[t]{0.268\columnwidth}\raggedright\strut
\textbf{Autor}\strut
\end{minipage} & \begin{minipage}[t]{0.76\columnwidth}\raggedright\strut
David Miguel Lozano\strut
\end{minipage}\tabularnewline
\begin{minipage}[t]{0.268\columnwidth}\raggedright\strut
\textbf{Requisitos asociados}\strut
\end{minipage} & \begin{minipage}[t]{0.76\columnwidth}\raggedright\strut
RF-1.5, RF-1.5.1\strut
\end{minipage}\tabularnewline
\begin{minipage}[t]{0.268\columnwidth}\raggedright\strut
\textbf{Descripción}\strut
\end{minipage} & \begin{minipage}[t]{0.76\columnwidth}\raggedright\strut
Permite al usuario visualizar toda la información relativa a un
determinado colmenar existente.\strut
\end{minipage}\tabularnewline
\begin{minipage}[t]{0.268\columnwidth}\raggedright\strut
\textbf{Precondición}\strut
\end{minipage} & \begin{minipage}[t]{0.76\columnwidth}\raggedright\strut
La base de datos se encuentra disponible.

El colmenar a visualizar existe.\strut
\end{minipage}\tabularnewline
\begin{minipage}[t]{0.268\columnwidth}\raggedright\strut
\textbf{Acciones}\strut
\end{minipage} & \begin{minipage}[t]{0.76\columnwidth}\raggedright\strut
\begin{enumerate}
\def\labelenumi{\arabic{enumi}.}
\tightlist
\item
  El usuario selecciona un colmenar para visualizar.
\item
  Se obtienen los datos del colmenar de la base de datos (incluidas sus
  colmenas).
\item
  Se actualiza su información meteorológica si no se dispone de esta o
  la que se dispone es de hace más de 15 minutos.
\item
  Se muestra una lista con sus colmenas.
\item
  Se muestra la información general del colmenar (localización, número
  de colmenas, última revisión y notas).
\item
  Se muestra la información meteorológica en detalle.
\end{enumerate}\strut
\end{minipage}\tabularnewline
\begin{minipage}[t]{0.268\columnwidth}\raggedright\strut
\textbf{Postcondición}\strut
\end{minipage} & \begin{minipage}[t]{0.76\columnwidth}\raggedright\strut
-\strut
\end{minipage}\tabularnewline
\begin{minipage}[t]{0.268\columnwidth}\raggedright\strut
\textbf{Excepciones}\strut
\end{minipage} & \begin{minipage}[t]{0.76\columnwidth}\raggedright\strut
\begin{itemize}
\tightlist
\item
  Error al cargar colmenar (mensaje).
\item
  No existe conexión a internet (mensaje).
\item
  Error al recuperar la información meteorológica (mensaje).
\end{itemize}\strut
\end{minipage}\tabularnewline
\begin{minipage}[t]{0.268\columnwidth}\raggedright\strut
\textbf{Importancia}\strut
\end{minipage} & \begin{minipage}[t]{0.76\columnwidth}\raggedright\strut
Alta\strut
\end{minipage}\tabularnewline
\bottomrule
\caption{CU-06 Ver colmenar.}
\end{longtable}

\begin{longtable}[H]{@{}ll@{}}
\toprule
\begin{minipage}[b]{0.26\columnwidth}\raggedright\strut
\textbf{CU-07}\strut
\end{minipage} & \begin{minipage}[b]{0.68\columnwidth}\raggedright\strut
\textbf{Obtener localización}\strut
\end{minipage}\tabularnewline
\midrule
\endhead
\begin{minipage}[t]{0.26\columnwidth}\raggedright\strut
\textbf{Versión}\strut
\end{minipage} & \begin{minipage}[t]{0.68\columnwidth}\raggedright\strut
1.0\strut
\end{minipage}\tabularnewline
\begin{minipage}[t]{0.26\columnwidth}\raggedright\strut
\textbf{Autor}\strut
\end{minipage} & \begin{minipage}[t]{0.68\columnwidth}\raggedright\strut
David Miguel Lozano\strut
\end{minipage}\tabularnewline
\begin{minipage}[t]{0.26\columnwidth}\raggedright\strut
\textbf{Requisitos asociados}\strut
\end{minipage} & \begin{minipage}[t]{0.68\columnwidth}\raggedright\strut
RF-1.1.1, RF-1.2.1\strut
\end{minipage}\tabularnewline
\begin{minipage}[t]{0.26\columnwidth}\raggedright\strut
\textbf{Descripción}\strut
\end{minipage} & \begin{minipage}[t]{0.68\columnwidth}\raggedright\strut
Permite obtener la localización actual del usuario.\strut
\end{minipage}\tabularnewline
\begin{minipage}[t]{0.26\columnwidth}\raggedright\strut
\textbf{Precondición}\strut
\end{minipage} & \begin{minipage}[t]{0.68\columnwidth}\raggedright\strut
Se poseen permisos de acceso a la localización.\strut
\end{minipage}\tabularnewline
\begin{minipage}[t]{0.26\columnwidth}\raggedright\strut
\textbf{Acciones}\strut
\end{minipage} & \begin{minipage}[t]{0.68\columnwidth}\raggedright\strut
\begin{enumerate}
\def\labelenumi{\arabic{enumi}.}
\tightlist
\item
  El usuario selecciona obtener localización actual.
\item
  La aplicación se conecta al servicio de localización.
\item
  El servicio de localización va devolviendo ubicaciones, cada vez más
  precisas.
\item
  Cuando el usuario considera la localización suficientemente buena,
  vuelve a presionar el botón de localización para detener la
  localización. Si no lo hace, se detendrá automáticamente al cambiar de
  actividad.
\item
  Se devuelve la localización obtenida.
\end{enumerate}\strut
\end{minipage}\tabularnewline
\begin{minipage}[t]{0.26\columnwidth}\raggedright\strut
\textbf{Postcondición}\strut
\end{minipage} & \begin{minipage}[t]{0.68\columnwidth}\raggedright\strut
Las coordenadas devueltas son válidas.\strut
\end{minipage}\tabularnewline
\begin{minipage}[t]{0.26\columnwidth}\raggedright\strut
\textbf{Excepciones}\strut
\end{minipage} & \begin{minipage}[t]{0.68\columnwidth}\raggedright\strut
\begin{itemize}
\tightlist
\item
  No se poseen permisos de localización (solicitar).
\item
  Error de conexión con el GPS (mensaje).
\end{itemize}\strut
\end{minipage}\tabularnewline
\begin{minipage}[t]{0.26\columnwidth}\raggedright\strut
\textbf{Importancia}\strut
\end{minipage} & \begin{minipage}[t]{0.68\columnwidth}\raggedright\strut
Alta\strut
\end{minipage}\tabularnewline
\bottomrule
\caption{CU-07 Obtener localización.}
\end{longtable}

\begin{longtable}[H]{@{}ll@{}}
\toprule
\begin{minipage}[b]{0.20\columnwidth}\raggedright\strut
\textbf{CU-08}\strut
\end{minipage} & \begin{minipage}[b]{0.74\columnwidth}\raggedright\strut
\textbf{Obtener información meteorológica}\strut
\end{minipage}\tabularnewline
\midrule
\endhead
\begin{minipage}[t]{0.20\columnwidth}\raggedright\strut
\textbf{Versión}\strut
\end{minipage} & \begin{minipage}[t]{0.74\columnwidth}\raggedright\strut
1.0\strut
\end{minipage}\tabularnewline
\begin{minipage}[t]{0.20\columnwidth}\raggedright\strut
\textbf{Autor}\strut
\end{minipage} & \begin{minipage}[t]{0.74\columnwidth}\raggedright\strut
David Miguel Lozano\strut
\end{minipage}\tabularnewline
\begin{minipage}[t]{0.20\columnwidth}\raggedright\strut
\textbf{Requisitos asociados}\strut
\end{minipage} & \begin{minipage}[t]{0.74\columnwidth}\raggedright\strut
RF-1.4.1, RF-1.5.1\strut
\end{minipage}\tabularnewline
\begin{minipage}[t]{0.20\columnwidth}\raggedright\strut
\textbf{Descripción}\strut
\end{minipage} & \begin{minipage}[t]{0.74\columnwidth}\raggedright\strut
Permite obtener la información meteorológica actual en un determinado
colmenar.\strut
\end{minipage}\tabularnewline
\begin{minipage}[t]{0.20\columnwidth}\raggedright\strut
\textbf{Precondición}\strut
\end{minipage} & \begin{minipage}[t]{0.74\columnwidth}\raggedright\strut
Se poseen permisos de acceso a internet.

El colmenar existe y posee localización.\strut
\end{minipage}\tabularnewline
\begin{minipage}[t]{0.20\columnwidth}\raggedright\strut
\textbf{Acciones}\strut
\end{minipage} & \begin{minipage}[t]{0.74\columnwidth}\raggedright\strut
\begin{enumerate}
\def\labelenumi{\arabic{enumi}.}
\tightlist
\item
  El sistema ejecuta la orden de actualizar información meteorológica
  para un determinado colmenar.
\item
  Se obtiene la ubicación del colmenar de la base de datos.
\item
  Se realiza una consulta a la API de \emph{OpenWeatherMap}.
\item
  Se procesan los datos recibidos.
\item
  Se devuelven los datos recibidos.
\end{enumerate}\strut
\end{minipage}\tabularnewline
\begin{minipage}[t]{0.20\columnwidth}\raggedright\strut
\textbf{Postcondición}\strut
\end{minipage} & \begin{minipage}[t]{0.74\columnwidth}\raggedright\strut
La información meteorológica devuelta es válida.\strut
\end{minipage}\tabularnewline
\begin{minipage}[t]{0.20\columnwidth}\raggedright\strut
\textbf{Excepciones}\strut
\end{minipage} & \begin{minipage}[t]{0.74\columnwidth}\raggedright\strut
\begin{itemize}
\tightlist
\item
  No se poseen permisos de internet (solicitar).
\item
  El colmenar no tiene localización (ignorar petición).
\end{itemize}\strut
\end{minipage}\tabularnewline
\begin{minipage}[t]{0.20\columnwidth}\raggedright\strut
\textbf{Importancia}\strut
\end{minipage} & \begin{minipage}[t]{0.74\columnwidth}\raggedright\strut
Alta\strut
\end{minipage}\tabularnewline
\bottomrule
\caption{CU-08 Obtener información meteorológica.}
\end{longtable}

\begin{longtable}[H]{@{}ll@{}}
\toprule
\begin{minipage}[b]{0.21\columnwidth}\raggedright\strut
\textbf{CU-09}\strut
\end{minipage} & \begin{minipage}[b]{0.73\columnwidth}\raggedright\strut
\textbf{Gestión de colmenas}\strut
\end{minipage}\tabularnewline
\midrule
\endhead
\begin{minipage}[t]{0.21\columnwidth}\raggedright\strut
\textbf{Versión}\strut
\end{minipage} & \begin{minipage}[t]{0.73\columnwidth}\raggedright\strut
1.0\strut
\end{minipage}\tabularnewline
\begin{minipage}[t]{0.21\columnwidth}\raggedright\strut
\textbf{Autor}\strut
\end{minipage} & \begin{minipage}[t]{0.73\columnwidth}\raggedright\strut
David Miguel Lozano\strut
\end{minipage}\tabularnewline
\begin{minipage}[t]{0.21\columnwidth}\raggedright\strut
\textbf{Requisitos asociados}\strut
\end{minipage} & \begin{minipage}[t]{0.73\columnwidth}\raggedright\strut
RF-2, RF-2.1, RF-2.2, RF-2.3, RF-2.4, RF-2.5\strut
\end{minipage}\tabularnewline
\begin{minipage}[t]{0.21\columnwidth}\raggedright\strut
\textbf{Descripción}\strut
\end{minipage} & \begin{minipage}[t]{0.73\columnwidth}\raggedright\strut
Permite al usuario gestionar las colmenas de un determinado
colmenar.\strut
\end{minipage}\tabularnewline
\begin{minipage}[t]{0.21\columnwidth}\raggedright\strut
\textbf{Precondición}\strut
\end{minipage} & \begin{minipage}[t]{0.73\columnwidth}\raggedright\strut
La base de datos se encuentra disponible.

El colmenar existe.\strut
\end{minipage}\tabularnewline
\begin{minipage}[t]{0.21\columnwidth}\raggedright\strut
\textbf{Acciones}\strut
\end{minipage} & \begin{minipage}[t]{0.73\columnwidth}\raggedright\strut
\begin{enumerate}
\def\labelenumi{\arabic{enumi}.}
\tightlist
\item
  El usuario entra en la vista detalle de un colmenar.
\item
  Se listan todas las colmenas.
\item
  Por cada colmena se da la opción de ver detalle, editar o eliminar.
\item
  Se muestra un botón para añadir una colmena.
\end{enumerate}\strut
\end{minipage}\tabularnewline
\begin{minipage}[t]{0.21\columnwidth}\raggedright\strut
\textbf{Postcondición}\strut
\end{minipage} & \begin{minipage}[t]{0.73\columnwidth}\raggedright\strut
El número de colmenas listado es igual al número de colmenas de ese
colmenar en la base de datos.\strut
\end{minipage}\tabularnewline
\begin{minipage}[t]{0.21\columnwidth}\raggedright\strut
\textbf{Excepciones}\strut
\end{minipage} & \begin{minipage}[t]{0.73\columnwidth}\raggedright\strut
\begin{itemize}
\tightlist
\item
  Error al cargar colmenas (mensaje).
\item
  No existe ninguna colmena (vista especial).
\end{itemize}\strut
\end{minipage}\tabularnewline
\begin{minipage}[t]{0.21\columnwidth}\raggedright\strut
\textbf{Importancia}\strut
\end{minipage} & \begin{minipage}[t]{0.73\columnwidth}\raggedright\strut
Alta\strut
\end{minipage}\tabularnewline
\bottomrule
\caption{CU-09 Gestión de colmenas.}
\end{longtable}

\begin{longtable}[H]{@{}ll@{}}
\toprule
\begin{minipage}[b]{0.269\columnwidth}\raggedright\strut
\textbf{CU-10}\strut
\end{minipage} & \begin{minipage}[b]{0.75\columnwidth}\raggedright\strut
\textbf{Añadir colmena}\strut
\end{minipage}\tabularnewline
\midrule
\endhead
\begin{minipage}[t]{0.269\columnwidth}\raggedright\strut
\textbf{Versión}\strut
\end{minipage} & \begin{minipage}[t]{0.75\columnwidth}\raggedright\strut
1.0\strut
\end{minipage}\tabularnewline
\begin{minipage}[t]{0.269\columnwidth}\raggedright\strut
\textbf{Autor}\strut
\end{minipage} & \begin{minipage}[t]{0.75\columnwidth}\raggedright\strut
David Miguel Lozano\strut
\end{minipage}\tabularnewline
\begin{minipage}[t]{0.269\columnwidth}\raggedright\strut
\textbf{Requisitos asociados}\strut
\end{minipage} & \begin{minipage}[t]{0.75\columnwidth}\raggedright\strut
RF-2.1\strut
\end{minipage}\tabularnewline
\begin{minipage}[t]{0.269\columnwidth}\raggedright\strut
\textbf{Descripción}\strut
\end{minipage} & \begin{minipage}[t]{0.75\columnwidth}\raggedright\strut
Permite al usuario añadir una nueva colmena.\strut
\end{minipage}\tabularnewline
\begin{minipage}[t]{0.269\columnwidth}\raggedright\strut
\textbf{Precondición}\strut
\end{minipage} & \begin{minipage}[t]{0.75\columnwidth}\raggedright\strut
La base de datos se encuentra disponible.

El colmenar existe.\strut
\end{minipage}\tabularnewline
\begin{minipage}[t]{0.269\columnwidth}\raggedright\strut
\textbf{Acciones}\strut
\end{minipage} & \begin{minipage}[t]{0.75\columnwidth}\raggedright\strut
\begin{enumerate}
\def\labelenumi{\arabic{enumi}.}
\tightlist
\item
  El usuario presiona en el botón de añadir colmena.
\item
  Se muestra el formulario para introducir los datos de la colmena.
\item
  El usuario introduce el nombre.
\item
  El usuario introduce notas sobre el colmenar (opcional).
\item
  El usuario pulsa el botón de aceptar.
\item
  Si no hay ningún error, se guarda una nueva colmena con los datos
  introducidos y se asocia al colmenar.
\item
  Volver a Gestión de colmenas.
\end{enumerate}\strut
\end{minipage}\tabularnewline
\begin{minipage}[t]{0.269\columnwidth}\raggedright\strut
\textbf{Postcondición}\strut
\end{minipage} & \begin{minipage}[t]{0.75\columnwidth}\raggedright\strut
Existe una colmena más para ese colmenar en la base de datos.\strut
\end{minipage}\tabularnewline
\begin{minipage}[t]{0.269\columnwidth}\raggedright\strut
\textbf{Excepciones}\strut
\end{minipage} & \begin{minipage}[t]{0.75\columnwidth}\raggedright\strut
\begin{itemize}
\tightlist
\item
  Error al guardar colmena (mensaje).
\item
  No se ha introducido nombre de la colmena (resaltar).
\end{itemize}\strut
\end{minipage}\tabularnewline
\begin{minipage}[t]{0.269\columnwidth}\raggedright\strut
\textbf{Importancia}\strut
\end{minipage} & \begin{minipage}[t]{0.75\columnwidth}\raggedright\strut
Alta\strut
\end{minipage}\tabularnewline
\bottomrule
\caption{CU-10 Añadir colmena.}
\end{longtable}

\begin{longtable}[H]{@{}ll@{}}
\toprule
\begin{minipage}[b]{0.26\columnwidth}\raggedright\strut
\textbf{CU-11}\strut
\end{minipage} & \begin{minipage}[b]{0.68\columnwidth}\raggedright\strut
\textbf{Editar colmena}\strut
\end{minipage}\tabularnewline
\midrule
\endhead
\begin{minipage}[t]{0.26\columnwidth}\raggedright\strut
\textbf{Versión}\strut
\end{minipage} & \begin{minipage}[t]{0.68\columnwidth}\raggedright\strut
1.0\strut
\end{minipage}\tabularnewline
\begin{minipage}[t]{0.26\columnwidth}\raggedright\strut
\textbf{Autor}\strut
\end{minipage} & \begin{minipage}[t]{0.68\columnwidth}\raggedright\strut
David Miguel Lozano\strut
\end{minipage}\tabularnewline
\begin{minipage}[t]{0.26\columnwidth}\raggedright\strut
\textbf{Requisitos asociados}\strut
\end{minipage} & \begin{minipage}[t]{0.68\columnwidth}\raggedright\strut
RF-2.2\strut
\end{minipage}\tabularnewline
\begin{minipage}[t]{0.26\columnwidth}\raggedright\strut
\textbf{Descripción}\strut
\end{minipage} & \begin{minipage}[t]{0.68\columnwidth}\raggedright\strut
Permite al usuario editar una colmena ya existente.\strut
\end{minipage}\tabularnewline
\begin{minipage}[t]{0.26\columnwidth}\raggedright\strut
\textbf{Precondición}\strut
\end{minipage} & \begin{minipage}[t]{0.68\columnwidth}\raggedright\strut
La base de datos se encuentra disponible.

El colmenar existe.\strut
\end{minipage}\tabularnewline
\begin{minipage}[t]{0.26\columnwidth}\raggedright\strut
\textbf{Acciones}\strut
\end{minipage} & \begin{minipage}[t]{0.68\columnwidth}\raggedright\strut
\begin{enumerate}
\def\labelenumi{\arabic{enumi}.}
\tightlist
\item
  El usuario selecciona una colmena para editar.
\item
  Se obtienen los datos de la colmena de la base de datos.
\item
  Se rellena el formulario de edición con los datos del colmenar.
\item
  El usuario edita alguno de los campos.
\item
  El usuario pulsa el botón aceptar.
\item
  Si no hay ningún error, se actualiza la colmena en la base de datos.
\end{enumerate}\strut
\end{minipage}\tabularnewline
\begin{minipage}[t]{0.26\columnwidth}\raggedright\strut
\textbf{Postcondición}\strut
\end{minipage} & \begin{minipage}[t]{0.68\columnwidth}\raggedright\strut
La información de la colmena en la base de datos ha sido
actualizada.\strut
\end{minipage}\tabularnewline
\begin{minipage}[t]{0.26\columnwidth}\raggedright\strut
\textbf{Excepciones}\strut
\end{minipage} & \begin{minipage}[t]{0.68\columnwidth}\raggedright\strut
\begin{itemize}
\tightlist
\item
  Error al guardar colmena (mensaje).
\item
  No se ha introducido nombre de la colmena (resaltar).
\end{itemize}\strut
\end{minipage}\tabularnewline
\begin{minipage}[t]{0.26\columnwidth}\raggedright\strut
\textbf{Importancia}\strut
\end{minipage} & \begin{minipage}[t]{0.68\columnwidth}\raggedright\strut
Alta\strut
\end{minipage}\tabularnewline
\bottomrule
\caption{CU-11 Editar colmena.}
\end{longtable}

\begin{longtable}[]{@{}ll@{}}
\toprule
\begin{minipage}[b]{0.29\columnwidth}\raggedright\strut
\textbf{CU-12}\strut
\end{minipage} & \begin{minipage}[b]{0.65\columnwidth}\raggedright\strut
\textbf{Eliminar colmena}\strut
\end{minipage}\tabularnewline
\midrule
\endhead
\begin{minipage}[t]{0.29\columnwidth}\raggedright\strut
\textbf{Versión}\strut
\end{minipage} & \begin{minipage}[t]{0.65\columnwidth}\raggedright\strut
1.0\strut
\end{minipage}\tabularnewline
\begin{minipage}[t]{0.29\columnwidth}\raggedright\strut
\textbf{Autor}\strut
\end{minipage} & \begin{minipage}[t]{0.65\columnwidth}\raggedright\strut
David Miguel Lozano\strut
\end{minipage}\tabularnewline
\begin{minipage}[t]{0.29\columnwidth}\raggedright\strut
\textbf{Requisitos asociados}\strut
\end{minipage} & \begin{minipage}[t]{0.65\columnwidth}\raggedright\strut
RF-2.3\strut
\end{minipage}\tabularnewline
\begin{minipage}[t]{0.29\columnwidth}\raggedright\strut
\textbf{Descripción}\strut
\end{minipage} & \begin{minipage}[t]{0.65\columnwidth}\raggedright\strut
Permite al usuario eliminar una colmena ya existente.\strut
\end{minipage}\tabularnewline
\begin{minipage}[t]{0.29\columnwidth}\raggedright\strut
\textbf{Precondición}\strut
\end{minipage} & \begin{minipage}[t]{0.65\columnwidth}\raggedright\strut
La base de datos se encuentra disponible.

El colmenar existe.

La colmena a eliminar existe.\strut
\end{minipage}\tabularnewline
\begin{minipage}[t]{0.29\columnwidth}\raggedright\strut
\textbf{Acciones}\strut
\end{minipage} & \begin{minipage}[t]{0.65\columnwidth}\raggedright\strut
\begin{enumerate}
\def\labelenumi{\arabic{enumi}.}
\tightlist
\item
  El usuario selecciona una colmena para eliminar.
\item
  Se eliminan los datos de esa colmena de la base de datos.
\item
  Se elimina la colmena de la vista.
\item
  Se informa al usuario.
\end{enumerate}\strut
\end{minipage}\tabularnewline
\begin{minipage}[t]{0.29\columnwidth}\raggedright\strut
\textbf{Postcondición}\strut
\end{minipage} & \begin{minipage}[t]{0.65\columnwidth}\raggedright\strut
Existe una colmena menos en ese colmenar en la base de datos.\strut
\end{minipage}\tabularnewline
\begin{minipage}[t]{0.29\columnwidth}\raggedright\strut
\textbf{Excepciones}\strut
\end{minipage} & \begin{minipage}[t]{0.65\columnwidth}\raggedright\strut
\begin{itemize}
\tightlist
\item
  Error al eliminar colmena (mensaje).
\end{itemize}\strut
\end{minipage}\tabularnewline
\begin{minipage}[t]{0.29\columnwidth}\raggedright\strut
\textbf{Importancia}\strut
\end{minipage} & \begin{minipage}[t]{0.65\columnwidth}\raggedright\strut
Alta\strut
\end{minipage}\tabularnewline
\bottomrule
\caption{CU-12 Eliminar colmena.}
\end{longtable}

\begin{longtable}[H]{@{}ll@{}}
\toprule
\begin{minipage}[b]{0.26\columnwidth}\raggedright\strut
\textbf{CU-13}\strut
\end{minipage} & \begin{minipage}[b]{0.68\columnwidth}\raggedright\strut
\textbf{Listar colmenas}\strut
\end{minipage}\tabularnewline
\midrule
\endhead
\begin{minipage}[t]{0.26\columnwidth}\raggedright\strut
\textbf{Versión}\strut
\end{minipage} & \begin{minipage}[t]{0.68\columnwidth}\raggedright\strut
1.0\strut
\end{minipage}\tabularnewline
\begin{minipage}[t]{0.26\columnwidth}\raggedright\strut
\textbf{Autor}\strut
\end{minipage} & \begin{minipage}[t]{0.68\columnwidth}\raggedright\strut
David Miguel Lozano\strut
\end{minipage}\tabularnewline
\begin{minipage}[t]{0.26\columnwidth}\raggedright\strut
\textbf{Requisitos asociados}\strut
\end{minipage} & \begin{minipage}[t]{0.68\columnwidth}\raggedright\strut
RF-2.4\strut
\end{minipage}\tabularnewline
\begin{minipage}[t]{0.26\columnwidth}\raggedright\strut
\textbf{Descripción}\strut
\end{minipage} & \begin{minipage}[t]{0.68\columnwidth}\raggedright\strut
Permite al usuario listar todas las colmenas de un determinado colmenar.
Por cada colmena se muestra el nombre y la fecha de la última
revisión.\strut
\end{minipage}\tabularnewline
\begin{minipage}[t]{0.26\columnwidth}\raggedright\strut
\textbf{Precondición}\strut
\end{minipage} & \begin{minipage}[t]{0.68\columnwidth}\raggedright\strut
La base de datos se encuentra disponible.

El colmenar existe.\strut
\end{minipage}\tabularnewline
\begin{minipage}[t]{0.26\columnwidth}\raggedright\strut
\textbf{Acciones}\strut
\end{minipage} & \begin{minipage}[t]{0.68\columnwidth}\raggedright\strut
\begin{enumerate}
\def\labelenumi{\arabic{enumi}.}
\tightlist
\item
  El usuario accede a Gestionar Colmenas de un determinado colmenar.
\item
  Se obtienen todas las colmenas de ese colmenar de la base de datos.
\item
  Se muestran la lista de colmenas. Cada elemento de la lista posee el
  nombre de la colmena y la fecha de la última revisión.
\end{enumerate}\strut
\end{minipage}\tabularnewline
\begin{minipage}[t]{0.26\columnwidth}\raggedright\strut
\textbf{Postcondición}\strut
\end{minipage} & \begin{minipage}[t]{0.68\columnwidth}\raggedright\strut
-\strut
\end{minipage}\tabularnewline
\begin{minipage}[t]{0.26\columnwidth}\raggedright\strut
\textbf{Excepciones}\strut
\end{minipage} & \begin{minipage}[t]{0.68\columnwidth}\raggedright\strut
\begin{itemize}
\tightlist
\item
  Error al cargar colmenas (mensaje).
\item
  No existen colmenas (vista especial).
\end{itemize}\strut
\end{minipage}\tabularnewline
\begin{minipage}[t]{0.26\columnwidth}\raggedright\strut
\textbf{Importancia}\strut
\end{minipage} & \begin{minipage}[t]{0.68\columnwidth}\raggedright\strut
Alta\strut
\end{minipage}\tabularnewline
\bottomrule
\caption{CU-13 Listar colmenas.}
\end{longtable}

\begin{longtable}[H]{@{}ll@{}}
\toprule
\begin{minipage}[b]{0.21\columnwidth}\raggedright\strut
\textbf{CU-14}\strut
\end{minipage} & \begin{minipage}[b]{0.73\columnwidth}\raggedright\strut
\textbf{Ver colmena}\strut
\end{minipage}\tabularnewline
\midrule
\endhead
\begin{minipage}[t]{0.21\columnwidth}\raggedright\strut
\textbf{Versión}\strut
\end{minipage} & \begin{minipage}[t]{0.73\columnwidth}\raggedright\strut
1.0\strut
\end{minipage}\tabularnewline
\begin{minipage}[t]{0.21\columnwidth}\raggedright\strut
\textbf{Autor}\strut
\end{minipage} & \begin{minipage}[t]{0.73\columnwidth}\raggedright\strut
David Miguel Lozano\strut
\end{minipage}\tabularnewline
\begin{minipage}[t]{0.21\columnwidth}\raggedright\strut
\textbf{Requisitos asociados}\strut
\end{minipage} & \begin{minipage}[t]{0.73\columnwidth}\raggedright\strut
RF-2.5\strut
\end{minipage}\tabularnewline
\begin{minipage}[t]{0.21\columnwidth}\raggedright\strut
\textbf{Descripción}\strut
\end{minipage} & \begin{minipage}[t]{0.73\columnwidth}\raggedright\strut
Permite al usuario visualizar toda la información relativa a una
determinada colmena existente.\strut
\end{minipage}\tabularnewline
\begin{minipage}[t]{0.21\columnwidth}\raggedright\strut
\textbf{Precondición}\strut
\end{minipage} & \begin{minipage}[t]{0.73\columnwidth}\raggedright\strut
La base de datos se encuentra disponible.

El colmenar existe.

La colmena a visualizar existe.\strut
\end{minipage}\tabularnewline
\begin{minipage}[t]{0.21\columnwidth}\raggedright\strut
\textbf{Acciones}\strut
\end{minipage} & \begin{minipage}[t]{0.73\columnwidth}\raggedright\strut
\begin{enumerate}
\def\labelenumi{\arabic{enumi}.}
\tightlist
\item
  El usuario selecciona una colmena de un determinado colmenar para
  visualizar.
\item
  Se obtienen los datos de la colmena de la base de datos (incluidas sus
  grabaciones).
\item
  Se muestra una lista con sus grabaciones.
\item
  Se muestra la información general de la colmena (última revisión y
  notas).
\end{enumerate}\strut
\end{minipage}\tabularnewline
\begin{minipage}[t]{0.21\columnwidth}\raggedright\strut
\textbf{Postcondición}\strut
\end{minipage} & \begin{minipage}[t]{0.73\columnwidth}\raggedright\strut
-\strut
\end{minipage}\tabularnewline
\begin{minipage}[t]{0.21\columnwidth}\raggedright\strut
\textbf{Excepciones}\strut
\end{minipage} & \begin{minipage}[t]{0.73\columnwidth}\raggedright\strut
\begin{itemize}
\tightlist
\item
  Error al cargar colmena (mensaje).
\end{itemize}\strut
\end{minipage}\tabularnewline
\begin{minipage}[t]{0.21\columnwidth}\raggedright\strut
\textbf{Importancia}\strut
\end{minipage} & \begin{minipage}[t]{0.73\columnwidth}\raggedright\strut
Alta\strut
\end{minipage}\tabularnewline
\bottomrule
\caption{CU-14 Ver colmena.}
\end{longtable}

\begin{longtable}[H]{@{}ll@{}}
\toprule
\begin{minipage}[b]{0.20\columnwidth}\raggedright\strut
\textbf{CU-15}\strut
\end{minipage} & \begin{minipage}[b]{0.74\columnwidth}\raggedright\strut
\textbf{Gestión de grabaciones}\strut
\end{minipage}\tabularnewline
\midrule
\endhead
\begin{minipage}[t]{0.20\columnwidth}\raggedright\strut
\textbf{Versión}\strut
\end{minipage} & \begin{minipage}[t]{0.74\columnwidth}\raggedright\strut
1.0\strut
\end{minipage}\tabularnewline
\begin{minipage}[t]{0.20\columnwidth}\raggedright\strut
\textbf{Autor}\strut
\end{minipage} & \begin{minipage}[t]{0.74\columnwidth}\raggedright\strut
David Miguel Lozano\strut
\end{minipage}\tabularnewline
\begin{minipage}[t]{0.20\columnwidth}\raggedright\strut
\textbf{Requisitos asociados}\strut
\end{minipage} & \begin{minipage}[t]{0.74\columnwidth}\raggedright\strut
RF-3, RF-3.1, RF-3.2, RF-3.3, RF-3.4\strut
\end{minipage}\tabularnewline
\begin{minipage}[t]{0.20\columnwidth}\raggedright\strut
\textbf{Descripción}\strut
\end{minipage} & \begin{minipage}[t]{0.74\columnwidth}\raggedright\strut
Permite al usuario gestionar las grabaciones de una determinada
colmena.\strut
\end{minipage}\tabularnewline
\begin{minipage}[t]{0.20\columnwidth}\raggedright\strut
\textbf{Precondición}\strut
\end{minipage} & \begin{minipage}[t]{0.74\columnwidth}\raggedright\strut
La base de datos se encuentra disponible.

El colmenar y la colmena existen.\strut
\end{minipage}\tabularnewline
\begin{minipage}[t]{0.20\columnwidth}\raggedright\strut
\textbf{Acciones}\strut
\end{minipage} & \begin{minipage}[t]{0.74\columnwidth}\raggedright\strut
\begin{enumerate}
\def\labelenumi{\arabic{enumi}.}
\tightlist
\item
  El usuario entra en la vista detalle de una colmena.
\item
  Se listan todas las grabaciones.
\item
  Por cada grabación se da la opción de ver detalle o eliminar.
\item
  Se muestra un botón para iniciar una nueva monitorización.
\end{enumerate}\strut
\end{minipage}\tabularnewline
\begin{minipage}[t]{0.20\columnwidth}\raggedright\strut
\textbf{Postcondición}\strut
\end{minipage} & \begin{minipage}[t]{0.74\columnwidth}\raggedright\strut
El número de grabaciones listado es igual al número de grabaciones de
esa colmena en la base de datos.\strut
\end{minipage}\tabularnewline
\begin{minipage}[t]{0.20\columnwidth}\raggedright\strut
\textbf{Excepciones}\strut
\end{minipage} & \begin{minipage}[t]{0.74\columnwidth}\raggedright\strut
\begin{itemize}
\tightlist
\item
  Error al cargar grabaciones (mensaje).
\item
  No existe ninguna grabación (vista especial).
\end{itemize}\strut
\end{minipage}\tabularnewline
\begin{minipage}[t]{0.20\columnwidth}\raggedright\strut
\textbf{Importancia}\strut
\end{minipage} & \begin{minipage}[t]{0.74\columnwidth}\raggedright\strut
Alta\strut
\end{minipage}\tabularnewline
\bottomrule
\caption{CU-15 Gestión de grabaciones.}
\end{longtable}

\begin{longtable}[H]{@{}ll@{}}
\toprule
\begin{minipage}[b]{0.26\columnwidth}\raggedright\strut
\textbf{CU-16}\strut
\end{minipage} & \begin{minipage}[b]{0.68\columnwidth}\raggedright\strut
\textbf{Añadir grabación}\strut
\end{minipage}\tabularnewline
\midrule
\endhead
\begin{minipage}[t]{0.26\columnwidth}\raggedright\strut
\textbf{Versión}\strut
\end{minipage} & \begin{minipage}[t]{0.68\columnwidth}\raggedright\strut
1.0\strut
\end{minipage}\tabularnewline
\begin{minipage}[t]{0.26\columnwidth}\raggedright\strut
\textbf{Autor}\strut
\end{minipage} & \begin{minipage}[t]{0.68\columnwidth}\raggedright\strut
David Miguel Lozano\strut
\end{minipage}\tabularnewline
\begin{minipage}[t]{0.26\columnwidth}\raggedright\strut
\textbf{Requisitos asociados}\strut
\end{minipage} & \begin{minipage}[t]{0.68\columnwidth}\raggedright\strut
RF-3.1\strut
\end{minipage}\tabularnewline
\begin{minipage}[t]{0.26\columnwidth}\raggedright\strut
\textbf{Descripción}\strut
\end{minipage} & \begin{minipage}[t]{0.68\columnwidth}\raggedright\strut
Permite añadir una nueva grabación a partir de los datos recogidos
durante la monitorización.\strut
\end{minipage}\tabularnewline
\begin{minipage}[t]{0.26\columnwidth}\raggedright\strut
\textbf{Precondición}\strut
\end{minipage} & \begin{minipage}[t]{0.68\columnwidth}\raggedright\strut
La base de datos se encuentra disponible.

El colmenar y la colmena existen.\strut
\end{minipage}\tabularnewline
\begin{minipage}[t]{0.26\columnwidth}\raggedright\strut
\textbf{Acciones}\strut
\end{minipage} & \begin{minipage}[t]{0.68\columnwidth}\raggedright\strut
\begin{enumerate}
\def\labelenumi{\arabic{enumi}.}
\tightlist
\item
  El usuario presiona el botón de finalizar monitorización.
\item
  Si no hay ningún error, se guarda una nueva grabación con los datos
  recogidos durante la monitorización (número de abejas e información
  meteorológica) y se asocia a la colmena.
\item
  Volver a Gestión de grabaciones.
\end{enumerate}\strut
\end{minipage}\tabularnewline
\begin{minipage}[t]{0.26\columnwidth}\raggedright\strut
\textbf{Postcondición}\strut
\end{minipage} & \begin{minipage}[t]{0.68\columnwidth}\raggedright\strut
Existe una grabación más para esa colmena en la base de datos.\strut
\end{minipage}\tabularnewline
\begin{minipage}[t]{0.26\columnwidth}\raggedright\strut
\textbf{Excepciones}\strut
\end{minipage} & \begin{minipage}[t]{0.68\columnwidth}\raggedright\strut
\begin{itemize}
\tightlist
\item
  Error al guardar grabación (mensaje).
\item
  Grabación demasiado corta (mensaje).
\end{itemize}\strut
\end{minipage}\tabularnewline
\begin{minipage}[t]{0.26\columnwidth}\raggedright\strut
\textbf{Importancia}\strut
\end{minipage} & \begin{minipage}[t]{0.68\columnwidth}\raggedright\strut
Alta\strut
\end{minipage}\tabularnewline
\bottomrule
\caption{CU-16 Añadir grabación.}
\end{longtable}

\begin{longtable}[H]{@{}ll@{}}
\toprule
\begin{minipage}[b]{0.28\columnwidth}\raggedright\strut
\textbf{CU-17}\strut
\end{minipage} & \begin{minipage}[b]{0.66\columnwidth}\raggedright\strut
\textbf{Eliminar grabación}\strut
\end{minipage}\tabularnewline
\midrule
\endhead
\begin{minipage}[t]{0.28\columnwidth}\raggedright\strut
\textbf{Versión}\strut
\end{minipage} & \begin{minipage}[t]{0.66\columnwidth}\raggedright\strut
1.0\strut
\end{minipage}\tabularnewline
\begin{minipage}[t]{0.28\columnwidth}\raggedright\strut
\textbf{Autor}\strut
\end{minipage} & \begin{minipage}[t]{0.66\columnwidth}\raggedright\strut
David Miguel Lozano\strut
\end{minipage}\tabularnewline
\begin{minipage}[t]{0.28\columnwidth}\raggedright\strut
\textbf{Requisitos asociados}\strut
\end{minipage} & \begin{minipage}[t]{0.66\columnwidth}\raggedright\strut
RF-3.2\strut
\end{minipage}\tabularnewline
\begin{minipage}[t]{0.28\columnwidth}\raggedright\strut
\textbf{Descripción}\strut
\end{minipage} & \begin{minipage}[t]{0.66\columnwidth}\raggedright\strut
Permite al usuario eliminar una grabación ya existente.\strut
\end{minipage}\tabularnewline
\begin{minipage}[t]{0.28\columnwidth}\raggedright\strut
\textbf{Precondición}\strut
\end{minipage} & \begin{minipage}[t]{0.66\columnwidth}\raggedright\strut
La base de datos se encuentra disponible.

El colmenar y la colmena existen.

La grabación a eliminar existe.\strut
\end{minipage}\tabularnewline
\begin{minipage}[t]{0.28\columnwidth}\raggedright\strut
\textbf{Acciones}\strut
\end{minipage} & \begin{minipage}[t]{0.66\columnwidth}\raggedright\strut
\begin{enumerate}
\def\labelenumi{\arabic{enumi}.}
\tightlist
\item
  El usuario selecciona una grabación para eliminar.
\item
  Se eliminan los datos de esa grabación de la base de datos.
\item
  Se elimina la grabación de la vista.
\item
  Se informa al usuario.
\end{enumerate}\strut
\end{minipage}\tabularnewline
\begin{minipage}[t]{0.28\columnwidth}\raggedright\strut
\textbf{Postcondición}\strut
\end{minipage} & \begin{minipage}[t]{0.66\columnwidth}\raggedright\strut
Existe una grabación menos en esa colmena en la base de datos.\strut
\end{minipage}\tabularnewline
\begin{minipage}[t]{0.28\columnwidth}\raggedright\strut
\textbf{Excepciones}\strut
\end{minipage} & \begin{minipage}[t]{0.66\columnwidth}\raggedright\strut
\begin{itemize}
\tightlist
\item
  Error al eliminar grabación (mensaje).
\end{itemize}\strut
\end{minipage}\tabularnewline
\begin{minipage}[t]{0.28\columnwidth}\raggedright\strut
\textbf{Importancia}\strut
\end{minipage} & \begin{minipage}[t]{0.66\columnwidth}\raggedright\strut
Alta\strut
\end{minipage}\tabularnewline
\bottomrule
\caption{CU-17 Eliminar grabación.}
\end{longtable}

\begin{longtable}[H]{@{}ll@{}}
\toprule
\begin{minipage}[b]{0.26\columnwidth}\raggedright\strut
\textbf{CU-18}\strut
\end{minipage} & \begin{minipage}[b]{0.68\columnwidth}\raggedright\strut
\textbf{Listar grabaciones}\strut
\end{minipage}\tabularnewline
\midrule
\endhead
\begin{minipage}[t]{0.26\columnwidth}\raggedright\strut
\textbf{Versión}\strut
\end{minipage} & \begin{minipage}[t]{0.68\columnwidth}\raggedright\strut
1.0\strut
\end{minipage}\tabularnewline
\begin{minipage}[t]{0.26\columnwidth}\raggedright\strut
\textbf{Autor}\strut
\end{minipage} & \begin{minipage}[t]{0.68\columnwidth}\raggedright\strut
David Miguel Lozano\strut
\end{minipage}\tabularnewline
\begin{minipage}[t]{0.26\columnwidth}\raggedright\strut
\textbf{Requisitos asociados}\strut
\end{minipage} & \begin{minipage}[t]{0.68\columnwidth}\raggedright\strut
RF-3.3\strut
\end{minipage}\tabularnewline
\begin{minipage}[t]{0.26\columnwidth}\raggedright\strut
\textbf{Descripción}\strut
\end{minipage} & \begin{minipage}[t]{0.68\columnwidth}\raggedright\strut
Permite al usuario listar todas las grabaciones de una determinada
colmena. Por cada grabación se muestra la fecha y una previsualización
de la actividad de vuelo.\strut
\end{minipage}\tabularnewline
\begin{minipage}[t]{0.26\columnwidth}\raggedright\strut
\textbf{Precondición}\strut
\end{minipage} & \begin{minipage}[t]{0.68\columnwidth}\raggedright\strut
La base de datos se encuentra disponible.

El colmenar y la colmena existen.\strut
\end{minipage}\tabularnewline
\begin{minipage}[t]{0.26\columnwidth}\raggedright\strut
\textbf{Acciones}\strut
\end{minipage} & \begin{minipage}[t]{0.68\columnwidth}\raggedright\strut
\begin{enumerate}
\def\labelenumi{\arabic{enumi}.}
\tightlist
\item
  El usuario accede a Gestionar Grabaciones de una determinada colmena.
\item
  Se obtienen todas las grabaciones de esa colmena de la base de datos.
\item
  Se muestran la lista de grabaciones. Cada elemento de la lista posee
  la fecha de la grabación y una previsualización de la actividad de
  vuelo.
\end{enumerate}\strut
\end{minipage}\tabularnewline
\begin{minipage}[t]{0.26\columnwidth}\raggedright\strut
\textbf{Postcondición}\strut
\end{minipage} & \begin{minipage}[t]{0.68\columnwidth}\raggedright\strut
-\strut
\end{minipage}\tabularnewline
\begin{minipage}[t]{0.26\columnwidth}\raggedright\strut
\textbf{Excepciones}\strut
\end{minipage} & \begin{minipage}[t]{0.68\columnwidth}\raggedright\strut
\begin{itemize}
\tightlist
\item
  Error al cargar grabaciones (mensaje).
\item
  No existen grabaciones (vista especial).
\end{itemize}\strut
\end{minipage}\tabularnewline
\begin{minipage}[t]{0.26\columnwidth}\raggedright\strut
\textbf{Importancia}\strut
\end{minipage} & \begin{minipage}[t]{0.68\columnwidth}\raggedright\strut
Alta\strut
\end{minipage}\tabularnewline
\bottomrule
\caption{CU-18 Listar grabaciones.}
\end{longtable}

\begin{longtable}[H]{@{}ll@{}}
\toprule
\begin{minipage}[b]{0.26\columnwidth}\raggedright\strut
\textbf{CU-19}\strut
\end{minipage} & \begin{minipage}[b]{0.68\columnwidth}\raggedright\strut
\textbf{Ver grabación}\strut
\end{minipage}\tabularnewline
\midrule
\endhead
\begin{minipage}[t]{0.26\columnwidth}\raggedright\strut
\textbf{Versión}\strut
\end{minipage} & \begin{minipage}[t]{0.68\columnwidth}\raggedright\strut
1.0\strut
\end{minipage}\tabularnewline
\begin{minipage}[t]{0.26\columnwidth}\raggedright\strut
\textbf{Autor}\strut
\end{minipage} & \begin{minipage}[t]{0.68\columnwidth}\raggedright\strut
David Miguel Lozano\strut
\end{minipage}\tabularnewline
\begin{minipage}[t]{0.26\columnwidth}\raggedright\strut
\textbf{Requisitos asociados}\strut
\end{minipage} & \begin{minipage}[t]{0.68\columnwidth}\raggedright\strut
RF-3.4\strut
\end{minipage}\tabularnewline
\begin{minipage}[t]{0.26\columnwidth}\raggedright\strut
\textbf{Descripción}\strut
\end{minipage} & \begin{minipage}[t]{0.68\columnwidth}\raggedright\strut
Permite al usuario visualizar toda la información (actividad de vuelo,
temperatura, precipitaciones y viento) relativa a una determinada
grabación existente.\strut
\end{minipage}\tabularnewline
\begin{minipage}[t]{0.26\columnwidth}\raggedright\strut
\textbf{Precondición}\strut
\end{minipage} & \begin{minipage}[t]{0.68\columnwidth}\raggedright\strut
La base de datos se encuentra disponible.

El colmenar y la colmena existen.

La grabación a visualizar existe.\strut
\end{minipage}\tabularnewline
\begin{minipage}[t]{0.26\columnwidth}\raggedright\strut
\textbf{Acciones}\strut
\end{minipage} & \begin{minipage}[t]{0.68\columnwidth}\raggedright\strut
\begin{enumerate}
\def\labelenumi{\arabic{enumi}.}
\tightlist
\item
  El usuario selecciona una grabación de una determinada colmena para
  visualizar.
\item
  Se obtienen los datos de la grabación de la base de datos (actividad
  de vuelo, temperatura, precipitaciones y viento).
\item
  Se muestra un gráfico con la actividad de vuelo.
\item
  Se muestra un gráfico con la evolución de la temperatura.
\item
  Se muestra un gráfico con la evolución de las precipitaciones.
\item
  Se muestra un gráfico con la evolución del viento.
\end{enumerate}\strut
\end{minipage}\tabularnewline
\begin{minipage}[t]{0.26\columnwidth}\raggedright\strut
\textbf{Postcondición}\strut
\end{minipage} & \begin{minipage}[t]{0.68\columnwidth}\raggedright\strut
-\strut
\end{minipage}\tabularnewline
\begin{minipage}[t]{0.26\columnwidth}\raggedright\strut
\textbf{Excepciones}\strut
\end{minipage} & \begin{minipage}[t]{0.68\columnwidth}\raggedright\strut
\begin{itemize}
\tightlist
\item
  Error al cargar grabación (mensaje).
\end{itemize}\strut
\end{minipage}\tabularnewline
\begin{minipage}[t]{0.26\columnwidth}\raggedright\strut
\textbf{Importancia}\strut
\end{minipage} & \begin{minipage}[t]{0.68\columnwidth}\raggedright\strut
Alta\strut
\end{minipage}\tabularnewline
\bottomrule
\caption{CU-19 Ver grabación.}
\end{longtable}

\begin{longtable}[H]{@{}ll@{}}
\toprule
\begin{minipage}[b]{0.269\columnwidth}\raggedright\strut
\textbf{CU-20}\strut
\end{minipage} & \begin{minipage}[b]{0.75\columnwidth}\raggedright\strut
\textbf{Monitorizar actividad de vuelo}\strut
\end{minipage}\tabularnewline
\midrule
\endhead
\begin{minipage}[t]{0.269\columnwidth}\raggedright\strut
\textbf{Versión}\strut
\end{minipage} & \begin{minipage}[t]{0.75\columnwidth}\raggedright\strut
1.0\strut
\end{minipage}\tabularnewline
\begin{minipage}[t]{0.269\columnwidth}\raggedright\strut
\textbf{Autor}\strut
\end{minipage} & \begin{minipage}[t]{0.75\columnwidth}\raggedright\strut
David Miguel Lozano\strut
\end{minipage}\tabularnewline
\begin{minipage}[t]{0.269\columnwidth}\raggedright\strut
\textbf{Requisitos asociados}\strut
\end{minipage} & \begin{minipage}[t]{0.75\columnwidth}\raggedright\strut
RF-4, RF-4.1, RF-4.2, RF-4.3\strut
\end{minipage}\tabularnewline
\begin{minipage}[t]{0.269\columnwidth}\raggedright\strut
\textbf{Descripción}\strut
\end{minipage} & \begin{minipage}[t]{0.75\columnwidth}\raggedright\strut
Permite al usuario monitorizar la actividad de vuelo de una colmena a
partir de una determinada parametrización.\strut
\end{minipage}\tabularnewline
\begin{minipage}[t]{0.269\columnwidth}\raggedright\strut
\textbf{Precondición}\strut
\end{minipage} & \begin{minipage}[t]{0.75\columnwidth}\raggedright\strut
Se poseen permisos de cámara.

La cámara se encuentra disponible.

El colmenar y la colmena existen.\strut
\end{minipage}\tabularnewline
\begin{minipage}[t]{0.269\columnwidth}\raggedright\strut
\textbf{Acciones}\strut
\end{minipage} & \begin{minipage}[t]{0.75\columnwidth}\raggedright\strut
\begin{enumerate}
\def\labelenumi{\arabic{enumi}.}
\tightlist
\item
  El usuario pulsa el botón de inicializar nueva monitorización.
\item
  Se muestra una previsualización de la salida del algoritmo.
\item
  Si el usuario presiona el botón de configurar:

  \begin{enumerate}
  \def\labelenumii{\alph{enumii}.}
  \tightlist
  \item
    Abrir ajustes.
  \item
    El usuario realiza los ajustes oportunos.
  \item
    Actualizar algoritmo y cámara con los ajustes.
  \item
    Volver a la previsualización.
  \end{enumerate}
\item
  Si el usuario presiona el botón de iniciar monitorización.

  \begin{enumerate}
  \def\labelenumii{\alph{enumii}.}
  \tightlist
  \item
    Se lanza el servicio de monitorización.
  \item
    Se realiza una cuenta atrás de 5 segundos antes de empezar a
    monitorizar.
  \item
    Se inicia la cámara.
  \item
    Se consumen los 10 primeros fotogramas para crear el modelo del
    fondo.
  \item
    Se comienza a monitorizar.
  \end{enumerate}
\item
  Por cada fotograma recibido:

  \begin{enumerate}
  \def\labelenumii{\alph{enumii}.}
  \tightlist
  \item
    Se convierte a escala de grises.
  \item
    Se aplica un desenfoque Gaussiano.
  \item
    Se aplica BackgroundSubtractorMOG2.
  \item
    Se aplican varias fases de erosión y dilatación.
  \item
    Se obtienen los contornos de las regiones en movimiento.
  \item
    Se contabilizan como abejas aquellos contornos que cumplen las
    condiciones.
  \item
    Se almacena el resultado.
  \end{enumerate}
\item
  Si el colmenar posee localización:

  \begin{enumerate}
  \def\labelenumii{\alph{enumii}.}
  \tightlist
  \item
    Cada 15 minutos, obtener información meteorológica.
  \item
    Guardarla en la base de datos asociada al colmenar.
  \end{enumerate}
\item
  Cuando se recibe la orden de finalizar:

  \begin{enumerate}
  \def\labelenumii{\alph{enumii}.}
  \tightlist
  \item
    Cerrar la cámara.
  \item
    Dejar de consultar información meteorológica.
  \item
    Devolver datos recolectados.
  \end{enumerate}
\end{enumerate}\strut
\end{minipage}\tabularnewline
\begin{minipage}[t]{0.269\columnwidth}\raggedright\strut
\textbf{Postcondición}\strut
\end{minipage} & \begin{minipage}[t]{0.75\columnwidth}\raggedright\strut
La grabación tiene más de 5 registros.\strut
\end{minipage}\tabularnewline
\begin{minipage}[t]{0.269\columnwidth}\raggedright\strut
\textbf{Excepciones}\strut
\end{minipage} & \begin{minipage}[t]{0.75\columnwidth}\raggedright\strut
\begin{itemize}
\tightlist
\item
  No se tienen permisos de cámara (solicitar).
\item
  Error de cámara (cancelar).
\item
  No existe conexión a internet (no obtener información meteorológica).
\item
  Error al obtener información meteorológica (ignorar).
\end{itemize}\strut
\end{minipage}\tabularnewline
\begin{minipage}[t]{0.269\columnwidth}\raggedright\strut
\textbf{Importancia}\strut
\end{minipage} & \begin{minipage}[t]{0.75\columnwidth}\raggedright\strut
Alta\strut
\end{minipage}\tabularnewline
\bottomrule
\caption{CU-20 Monitorizar actividad de vuelo.}
\end{longtable}

\begin{longtable}[H]{@{}ll@{}}
\toprule
\begin{minipage}[b]{0.26\columnwidth}\raggedright\strut
\textbf{CU-21}\strut
\end{minipage} & \begin{minipage}[b]{0.68\columnwidth}\raggedright\strut
\textbf{Previsualización del algoritmo}\strut
\end{minipage}\tabularnewline
\midrule
\endhead
\begin{minipage}[t]{0.26\columnwidth}\raggedright\strut
\textbf{Versión}\strut
\end{minipage} & \begin{minipage}[t]{0.68\columnwidth}\raggedright\strut
1.0\strut
\end{minipage}\tabularnewline
\begin{minipage}[t]{0.26\columnwidth}\raggedright\strut
\textbf{Autor}\strut
\end{minipage} & \begin{minipage}[t]{0.68\columnwidth}\raggedright\strut
David Miguel Lozano\strut
\end{minipage}\tabularnewline
\begin{minipage}[t]{0.26\columnwidth}\raggedright\strut
\textbf{Requisitos asociados}\strut
\end{minipage} & \begin{minipage}[t]{0.68\columnwidth}\raggedright\strut
RF-4.1\strut
\end{minipage}\tabularnewline
\begin{minipage}[t]{0.26\columnwidth}\raggedright\strut
\textbf{Descripción}\strut
\end{minipage} & \begin{minipage}[t]{0.68\columnwidth}\raggedright\strut
Permite al usuario previsualizar los resultados que está proporcionando
el algoritmo de conteo en tiempo real (visualización de los fotogramas
de entrada, la máscara de salida y el número de abejas contadas).\strut
\end{minipage}\tabularnewline
\begin{minipage}[t]{0.26\columnwidth}\raggedright\strut
\textbf{Precondición}\strut
\end{minipage} & \begin{minipage}[t]{0.68\columnwidth}\raggedright\strut
Se poseen permisos de cámara.

La cámara se encuentra disponible.

El colmenar y la colmena existen.\strut
\end{minipage}\tabularnewline
\begin{minipage}[t]{0.26\columnwidth}\raggedright\strut
\textbf{Acciones}\strut
\end{minipage} & \begin{minipage}[t]{0.68\columnwidth}\raggedright\strut
\begin{enumerate}
\def\labelenumi{\arabic{enumi}.}
\tightlist
\item
  El usuario pulsa el botón de inicializar nueva monitorización.
\item
  Se muestra en tiempo real los fotogramas (bien los de entrada del
  algoritmo o los de salida).
\item
  Se muestra el número de abejas que contabiliza en cada fotograma
  analizado.
\item
  Si el usuario modifica algún ajuste, se actualiza en la
  previsualización.
\end{enumerate}\strut
\end{minipage}\tabularnewline
\begin{minipage}[t]{0.26\columnwidth}\raggedright\strut
\textbf{Postcondición}\strut
\end{minipage} & \begin{minipage}[t]{0.68\columnwidth}\raggedright\strut
-\strut
\end{minipage}\tabularnewline
\begin{minipage}[t]{0.26\columnwidth}\raggedright\strut
\textbf{Excepciones}\strut
\end{minipage} & \begin{minipage}[t]{0.68\columnwidth}\raggedright\strut
\begin{itemize}
\tightlist
\item
  No se tienen permisos de cámara (solicitar).
\item
  Error de cámara (cancelar).
\end{itemize}\strut
\end{minipage}\tabularnewline
\begin{minipage}[t]{0.26\columnwidth}\raggedright\strut
\textbf{Importancia}\strut
\end{minipage} & \begin{minipage}[t]{0.68\columnwidth}\raggedright\strut
Alta\strut
\end{minipage}\tabularnewline
\bottomrule
\caption{CU-21 Previsualización del algoritmo.}
\end{longtable}

\begin{longtable}[H]{@{}ll@{}}
\toprule
\begin{minipage}[b]{0.26\columnwidth}\raggedright\strut
\textbf{CU-22}\strut
\end{minipage} & \begin{minipage}[b]{0.68\columnwidth}\raggedright\strut
\textbf{Configuración de la monitorización}\strut
\end{minipage}\tabularnewline
\midrule
\endhead
\begin{minipage}[t]{0.26\columnwidth}\raggedright\strut
\textbf{Versión}\strut
\end{minipage} & \begin{minipage}[t]{0.68\columnwidth}\raggedright\strut
1.0\strut
\end{minipage}\tabularnewline
\begin{minipage}[t]{0.26\columnwidth}\raggedright\strut
\textbf{Autor}\strut
\end{minipage} & \begin{minipage}[t]{0.68\columnwidth}\raggedright\strut
David Miguel Lozano\strut
\end{minipage}\tabularnewline
\begin{minipage}[t]{0.26\columnwidth}\raggedright\strut
\textbf{Requisitos asociados}\strut
\end{minipage} & \begin{minipage}[t]{0.68\columnwidth}\raggedright\strut
RF-4.2\strut
\end{minipage}\tabularnewline
\begin{minipage}[t]{0.26\columnwidth}\raggedright\strut
\textbf{Descripción}\strut
\end{minipage} & \begin{minipage}[t]{0.68\columnwidth}\raggedright\strut
Permite al usuario configurar todos los parámetros relativos a la
monitorización (parámetros del algoritmo y parámetros de la
cámara).\strut
\end{minipage}\tabularnewline
\begin{minipage}[t]{0.26\columnwidth}\raggedright\strut
\textbf{Precondición}\strut
\end{minipage} & \begin{minipage}[t]{0.68\columnwidth}\raggedright\strut
Se poseen permisos de cámara.

La cámara se encuentra disponible.

El colmenar y la colmena existen.\strut
\end{minipage}\tabularnewline
\begin{minipage}[t]{0.26\columnwidth}\raggedright\strut
\textbf{Acciones}\strut
\end{minipage} & \begin{minipage}[t]{0.68\columnwidth}\raggedright\strut
\begin{enumerate}
\def\labelenumi{\arabic{enumi}.}
\tightlist
\item
  El usuario se encuentra en la pantalla de previsualización y pulsa el
  botón de ajustes.
\item
  Se abre una ventana con los diferentes parámetros ajustables (mostrar
  salida o entrada del algoritmo, modificar tamaño de las regiones,
  ajustar áreas de una abeja, zoom y frecuencia de muestreo).
\item
  Cuando el usuario realiza alguna modificación, actualizar
  instantáneamente ese parámetro en la cámara o en el algoritmo.
\end{enumerate}\strut
\end{minipage}\tabularnewline
\begin{minipage}[t]{0.26\columnwidth}\raggedright\strut
\textbf{Postcondición}\strut
\end{minipage} & \begin{minipage}[t]{0.68\columnwidth}\raggedright\strut
-\strut
\end{minipage}\tabularnewline
\begin{minipage}[t]{0.26\columnwidth}\raggedright\strut
\textbf{Excepciones}\strut
\end{minipage} & \begin{minipage}[t]{0.68\columnwidth}\raggedright\strut
\begin{itemize}
\tightlist
\item
  No se tienen permisos de cámara (solicitar).
\item
  Error de cámara (cancelar).
\end{itemize}\strut
\end{minipage}\tabularnewline
\begin{minipage}[t]{0.26\columnwidth}\raggedright\strut
\textbf{Importancia}\strut
\end{minipage} & \begin{minipage}[t]{0.68\columnwidth}\raggedright\strut
Alta\strut
\end{minipage}\tabularnewline
\bottomrule
\caption{CU-22 Configuración de la monitorización.}
\end{longtable}

\begin{longtable}[H]{@{}ll@{}}
\toprule
\begin{minipage}[b]{0.26\columnwidth}\raggedright\strut
\textbf{CU-23}\strut
\end{minipage} & \begin{minipage}[b]{0.68\columnwidth}\raggedright\strut
\textbf{Configuración de la aplicación}\strut
\end{minipage}\tabularnewline
\midrule
\endhead
\begin{minipage}[t]{0.26\columnwidth}\raggedright\strut
\textbf{Versión}\strut
\end{minipage} & \begin{minipage}[t]{0.68\columnwidth}\raggedright\strut
1.0\strut
\end{minipage}\tabularnewline
\begin{minipage}[t]{0.26\columnwidth}\raggedright\strut
\textbf{Autor}\strut
\end{minipage} & \begin{minipage}[t]{0.68\columnwidth}\raggedright\strut
David Miguel Lozano\strut
\end{minipage}\tabularnewline
\begin{minipage}[t]{0.26\columnwidth}\raggedright\strut
\textbf{Requisitos asociados}\strut
\end{minipage} & \begin{minipage}[t]{0.68\columnwidth}\raggedright\strut
RF-5\strut
\end{minipage}\tabularnewline
\begin{minipage}[t]{0.26\columnwidth}\raggedright\strut
\textbf{Descripción}\strut
\end{minipage} & \begin{minipage}[t]{0.68\columnwidth}\raggedright\strut
Permite al usuario configurar todos los parámetros disponibles en la
aplicación, como el idioma o las unidades meteorológicas o realizar
determinadas tareas de mantenimiento.\strut
\end{minipage}\tabularnewline
\begin{minipage}[t]{0.26\columnwidth}\raggedright\strut
\textbf{Precondición}\strut
\end{minipage} & \begin{minipage}[t]{0.68\columnwidth}\raggedright\strut
La base de datos se encuentra disponible.\strut
\end{minipage}\tabularnewline
\begin{minipage}[t]{0.26\columnwidth}\raggedright\strut
\textbf{Acciones}\strut
\end{minipage} & \begin{minipage}[t]{0.68\columnwidth}\raggedright\strut
\begin{enumerate}
\def\labelenumi{\arabic{enumi}.}
\tightlist
\item
  El usuario presiona el botón de ajustes de aplicación.
\item
  Se abre una ventana con los diferentes parámetros ajustables.
\item
  Si el usuario modifica cualquier parámetro, se hace efectiva la nueva
  configuración al instante.
\end{enumerate}\strut
\end{minipage}\tabularnewline
\begin{minipage}[t]{0.26\columnwidth}\raggedright\strut
\textbf{Postcondición}\strut
\end{minipage} & \begin{minipage}[t]{0.68\columnwidth}\raggedright\strut
-\strut
\end{minipage}\tabularnewline
\begin{minipage}[t]{0.26\columnwidth}\raggedright\strut
\textbf{Excepciones}\strut
\end{minipage} & \begin{minipage}[t]{0.68\columnwidth}\raggedright\strut
\begin{itemize}
\tightlist
\item
  Error al guardar configuración (mensaje).
\end{itemize}\strut
\end{minipage}\tabularnewline
\begin{minipage}[t]{0.26\columnwidth}\raggedright\strut
\textbf{Importancia}\strut
\end{minipage} & \begin{minipage}[t]{0.68\columnwidth}\raggedright\strut
Alta\strut
\end{minipage}\tabularnewline
\bottomrule
\caption{CU-23 Configuración de la aplicación.}
\end{longtable}

\begin{longtable}[H]{@{}ll@{}}
\toprule
\begin{minipage}[b]{0.20\columnwidth}\raggedright\strut
\textbf{CU-24}\strut
\end{minipage} & \begin{minipage}[b]{0.74\columnwidth}\raggedright\strut
\textbf{Ayuda de la aplicación}\strut
\end{minipage}\tabularnewline
\midrule
\endhead
\begin{minipage}[t]{0.20\columnwidth}\raggedright\strut
\textbf{Versión}\strut
\end{minipage} & \begin{minipage}[t]{0.74\columnwidth}\raggedright\strut
1.0\strut
\end{minipage}\tabularnewline
\begin{minipage}[t]{0.20\columnwidth}\raggedright\strut
\textbf{Autor}\strut
\end{minipage} & \begin{minipage}[t]{0.74\columnwidth}\raggedright\strut
David Miguel Lozano\strut
\end{minipage}\tabularnewline
\begin{minipage}[t]{0.20\columnwidth}\raggedright\strut
\textbf{Requisitos asociados}\strut
\end{minipage} & \begin{minipage}[t]{0.74\columnwidth}\raggedright\strut
RF-6\strut
\end{minipage}\tabularnewline
\begin{minipage}[t]{0.20\columnwidth}\raggedright\strut
\textbf{Descripción}\strut
\end{minipage} & \begin{minipage}[t]{0.74\columnwidth}\raggedright\strut
Permite al usuario obtener ayuda sobre cada una de las funcionalidades
de la aplicación.\strut
\end{minipage}\tabularnewline
\begin{minipage}[t]{0.20\columnwidth}\raggedright\strut
\textbf{Precondición}\strut
\end{minipage} & \begin{minipage}[t]{0.74\columnwidth}\raggedright\strut
Se dispone de permisos de internet.

Se dispone de conexión a internet.\strut
\end{minipage}\tabularnewline
\begin{minipage}[t]{0.20\columnwidth}\raggedright\strut
\textbf{Acciones}\strut
\end{minipage} & \begin{minipage}[t]{0.74\columnwidth}\raggedright\strut
\begin{enumerate}
\def\labelenumi{\arabic{enumi}.}
\tightlist
\item
  El usuario presiona el botón de ayuda de aplicación.
\item
  Se abre una ventana que carga una página web con la ayuda de la
  aplicación categorizada por acciones.
\end{enumerate}\strut
\end{minipage}\tabularnewline
\begin{minipage}[t]{0.20\columnwidth}\raggedright\strut
\textbf{Postcondición}\strut
\end{minipage} & \begin{minipage}[t]{0.74\columnwidth}\raggedright\strut
-\strut
\end{minipage}\tabularnewline
\begin{minipage}[t]{0.20\columnwidth}\raggedright\strut
\textbf{Excepciones}\strut
\end{minipage} & \begin{minipage}[t]{0.74\columnwidth}\raggedright\strut
\begin{itemize}
\tightlist
\item
  No se disponen de permisos de internet (solicitar),
\item
  No hay conexión a internet (mensaje).
\end{itemize}\strut
\end{minipage}\tabularnewline
\begin{minipage}[t]{0.20\columnwidth}\raggedright\strut
\textbf{Importancia}\strut
\end{minipage} & \begin{minipage}[t]{0.74\columnwidth}\raggedright\strut
Alta\strut
\end{minipage}\tabularnewline
\bottomrule
\caption{CU-24 Ayuda de la aplicación.}
\end{longtable}

\begin{longtable}[H]{@{}ll@{}}
\toprule
\begin{minipage}[b]{0.26\columnwidth}\raggedright\strut
\textbf{CU-25}\strut
\end{minipage} & \begin{minipage}[b]{0.68\columnwidth}\raggedright\strut
\textbf{Información de la aplicación}\strut
\end{minipage}\tabularnewline
\midrule
\endhead
\begin{minipage}[t]{0.26\columnwidth}\raggedright\strut
\textbf{Versión}\strut
\end{minipage} & \begin{minipage}[t]{0.68\columnwidth}\raggedright\strut
1.0\strut
\end{minipage}\tabularnewline
\begin{minipage}[t]{0.26\columnwidth}\raggedright\strut
\textbf{Autor}\strut
\end{minipage} & \begin{minipage}[t]{0.68\columnwidth}\raggedright\strut
David Miguel Lozano\strut
\end{minipage}\tabularnewline
\begin{minipage}[t]{0.26\columnwidth}\raggedright\strut
\textbf{Requisitos asociados}\strut
\end{minipage} & \begin{minipage}[t]{0.68\columnwidth}\raggedright\strut
RF-7\strut
\end{minipage}\tabularnewline
\begin{minipage}[t]{0.26\columnwidth}\raggedright\strut
\textbf{Descripción}\strut
\end{minipage} & \begin{minipage}[t]{0.68\columnwidth}\raggedright\strut
Permite al usuario obtener información sobre la aplicación, compartirla
o enviar sugerencias.\strut
\end{minipage}\tabularnewline
\begin{minipage}[t]{0.26\columnwidth}\raggedright\strut
\textbf{Precondición}\strut
\end{minipage} & \begin{minipage}[t]{0.68\columnwidth}\raggedright\strut
-\strut
\end{minipage}\tabularnewline
\begin{minipage}[t]{0.26\columnwidth}\raggedright\strut
\textbf{Acciones}\strut
\end{minipage} & \begin{minipage}[t]{0.68\columnwidth}\raggedright\strut
\begin{enumerate}
\def\labelenumi{\arabic{enumi}.}
\tightlist
\item
  Si el usuario presiona el botón de compartir aplicación, se le
  muestran los diferentes medios soportados por el dispositivo para
  compartirla.
\item
  Si el usuario presiona sobre el botón de enviar comentarios, se abre
  la aplicación de email con la información del destinatario rellenada
  para que el usuario pueda enviar sus sugerencias.
\item
  Si el usuario presiona sobre el botón acerca de GoBees, se abre una
  ventana con información sobre la versión, autor, licencia, página web,
  historial de cambios y librerías utilizadas junto con sus licencias.
\end{enumerate}\strut
\end{minipage}\tabularnewline
\begin{minipage}[t]{0.26\columnwidth}\raggedright\strut
\textbf{Postcondición}\strut
\end{minipage} & \begin{minipage}[t]{0.68\columnwidth}\raggedright\strut
-\strut
\end{minipage}\tabularnewline
\begin{minipage}[t]{0.26\columnwidth}\raggedright\strut
\textbf{Excepciones}\strut
\end{minipage} & \begin{minipage}[t]{0.68\columnwidth}\raggedright\strut
-\strut
\end{minipage}\tabularnewline
\begin{minipage}[t]{0.26\columnwidth}\raggedright\strut
\textbf{Importancia}\strut
\end{minipage} & \begin{minipage}[t]{0.68\columnwidth}\raggedright\strut
Media\strut
\end{minipage}\tabularnewline
\bottomrule
\caption{CU-25 Información de la aplicación.}
\end{longtable}
