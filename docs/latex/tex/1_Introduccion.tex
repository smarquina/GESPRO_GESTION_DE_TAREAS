\capitulo{1}{Introducción}

Las abejas son una pieza clave en nuestro ecosistema. La producción de
alimentos a nivel mundial y la biodiversidad de nuestro planeta dependen
en gran medida de ellas. Son las encargadas de polinizar el 70\% de los
cultivos de comida, que suponen un 90\% de la alimentación humana \citep{art:bees_decline}. 
Sin embargo, la población mundial de abejas
está disminuyendo a pasos agigantados en los últimos años debido, entre
otras causas, al uso extendido de plaguicidas tóxicos, parásitos como la
varroa o la expansión del avispón asiático \citep{art:ccd}.

Actualmente los apicultores inspeccionan sus colmenares de forma manual.
Uno de los indicadores que más información les proporciona es la
actividad de vuelo de la colmena (número de abejas en vuelo a la entrada
de la colmena en un determinado instante) \citep{art:campbell2008}. Este
dato, junto con información previa de la colmena y conocimiento de las
condiciones locales, permite al apicultor conocer el estado de la
colmena con bastante seguridad, pudiendo determinar si esta necesita o
no una intervención.

La actividad de vuelo de una colmena varía dependiendo de múltiples
factores, tanto externos como internos a la colmena. Entre ellos se
encuentran la propia población de la colmena, las condiciones
meteorológicas, la presencia de enfermedades, parásitos o depredadores,
la exposición a tóxicos, la presencia de fuentes de néctar, etc. A pesar
de los numerosos factores que influyen en la actividad de vuelo, su
conocimiento es de gran ayuda para la toma de decisiones por parte del
apicultor. Ya que este posee información sobre la mayoría de los
factores necesarios para su interpretación.

La captación prolongada de la actividad de vuelo de forma manual es muy
costosa, tediosa y puede introducir una tasa de error elevada. Es por
esto que a lo largo de los años se haya intentado automatizar este
proceso de muy diversas maneras. Los primeros intentos se remontan a
principios del siglo XX, donde se desarrollaron contadores por contacto
eléctrico \citep{art:lundie1925}. Otros métodos posteriores se basan en
sensores de infrarrojos \citep{art:struye1994}, sensores capacitivos
\citep{art:campbell2005}, códigos de barras \citep{beebarcode} o incluso
en microchips acoplados a las abejas
\citep{art:decourtye_honeybee_2011}. En los últimos años, se han
desarrollado numerosos métodos basados en visión artificial
\citep{art:campbell2008,art:chiron2013a,art:chiron2013,art:tashakkori2015}.

Los métodos basados en contacto, sensores de infrarrojos o capacitivos
tienen el inconveniente de que es necesario realizar modificaciones en
la colmena, mientras que en los basados en códigos de barras o
microchips es necesario manipular las abejas directamente. Estos motivos
los convierten en métodos poco prácticos fuera del campo investigador.
Por el contrario, la visión artificial aporta un gran potencial, ya que
evita tener que realizar ningún tipo de modificación ni en el entorno,
ni en las abejas. Además, abre la puerta a la monitorización de nuevos
parámetros como la detección de enjambrazón (división de la colmena y
salida de un enjambre) o la detección de amenazas (avispones, abejaruco,
etc.).

Todos los métodos basados en visión artificial propuestos hasta la fecha
utilizan hardware específico con un coste elevado. En este trabajo se
propone un método de monitorización de la actividad de vuelo de una
colmena mediante la cámara de un \emph{smartphone} con Android.

El método propuesto podría revolucionar el campo de la monitorización de
la actividad de vuelo de colmenas, ya que lo hace accesible al gran
público. Ya no es necesario contar con costoso hardware, difícil de
instalar. Solamente es necesario disponer de un \emph{smartphone} con cámara y
la aplicación GoBees. Además, esta facilita la interpretación de los
datos, representándolos gráficamente y añadiendo información adicional
como la situación meteorológica. Permitiendo a los apicultores centrar
su atención donde realmente es necesaria.

\section{Estructura de la memoria}\label{estructura-de-la-memoria}

La memoria sigue la siguiente estructura:

\begin{itemize}
\tightlist
\item
  \textbf{Introducción:} breve descripción del problema a resolver y la
  solución propuesta. Estructura de la memoria y listado de materiales
  adjuntos.
\item
  \textbf{Objetivos del proyecto:} exposición de los objetivos que
  persigue el proyecto.
\item
  \textbf{Conceptos teóricos:} breve explicación de los conceptos
  teóricos clave para la comprensión de la solución propuesta.
\item
  \textbf{Técnicas y herramientas:} listado de técnicas metodológicas y
  herramientas utilizadas para gestión y desarrollo del proyecto.
\item
  \textbf{Aspectos relevantes del desarrollo:} exposición de aspectos
  destacables que tuvieron lugar durante la realización del proyecto.
\item
  \textbf{Trabajos relacionados:} estado del arte en el campo de la
  monitorización de la actividad de vuelo de colmenas y proyectos
  relacionados.
\item
  \textbf{Conclusiones y líneas de trabajo futuras:} conclusiones
  obtenidas tras la realización del proyecto y posibilidades de mejora o
  expansión de la solución aportada.
\end{itemize}

Junto a la memoria se proporcionan los siguientes anexos:

\begin{itemize}
\tightlist
\item
  \textbf{Plan del proyecto software:} planificación temporal y estudio
  de viabilidad del proyecto.
\item
  \textbf{Especificación de requisitos del software:} se describe la
  fase de análisis; los objetivos generales, el catálogo de requisitos
  del sistema y la especificación de requisitos funcionales y no
  funcionales.
\item
  \textbf{Especificación de diseño:} se describe la fase de diseño; el
  ámbito del software, el diseño de datos, el diseño procedimental y el
  diseño arquitectónico.
\item
  \textbf{Manual del programador:} recoge los aspectos más relevantes
  relacionados con el código fuente (estructura, compilación,
  instalación, ejecución, pruebas, etc.).
\item
  \textbf{Manual de usuario:} guía de usuario para el correcto manejo de
  la aplicación.
\end{itemize}

\section{Materiales adjuntos}\label{materiales-adjuntos}

Los materiales que se adjuntan con la memoria son: 

\begin{itemize}
\tightlist
\item
	Aplicación para Android GoBees.
\item
	Aplicación Java para el desarrollo del algoritmo.
\item	
	Aplicación Java para el etiquetado de fotogramas.
\item	
	JavaDoc.
\item	
	\emph{Dataset} de vídeos de prueba.
\end{itemize}

Además, los siguientes recursos están accesibles a través de internet:

\begin{itemize}
\tightlist
\item
  Página web del proyecto \citep{gobees:web}.
\item
  GoBees en Google Play \citep{gobees:play}.
\item
  Repositorio del proyecto \citep{gobees:repo}.
\item
  Repositorio de las herramientas desarrolladas para el proyecto \citep{gobees:prototipes}.
\end{itemize}
