\capitulo{6}{Trabajos relacionados}

Como se comentó en la introducción, los intentos de automatizar el
proceso de monitorización de la actividad de una colmena se remontan
hasta principios del siglo pasado. Sin embargo, no es hasta 2008 cuando
se introduce la visión artificial en este campo. A continuación, se
exponen los artículos científicos relacionados publicados hasta la
fecha, así como proyectos con objetivos similares.

\section{Artículos científicos}\label{artuxedculos-cientuxedficos}

\subsection{\emph{Video Monitoring of Honey Bee Colonies at the Hive
Entrance}}\label{video-monitoring-of-honey-bee-colonies-at-the-hive-entrance}

Se trata del primer artículo publicado sobre el tema (año 2008). Los
autores fueron Jason Campbell, Lily Mummert y Rahul Sukthankar del
\emph{Intel Research Pittsburgh}. En él proponen un método de visión
artificial para monitorizar las entradas y salidas de abejas en una
colmena, consiguiendo diferenciar las que entran de las que salen. Se
describen los desafíos técnicos que supuso y la solución a la que
llegaron finalmente \citep{art:campbell2008}.

\subsection{\emph{Detecting and tracking honeybees in 3D at the beehive
entrance using stereo
vision}}\label{detecting-and-tracking-honeybees-in-3d-at-the-beehive-entrance-using-stereo-vision}

En 2013, Guillaume Chiron, Petra Gomez-Krämer y Ménard Michel publicaron
un artículo en \emph{EURASIP Journal on Image and Video Processing,}
donde proponían un método para la monitorización de abejas a la entrada
de una colmena basado en un sistema de tiempo real con visión
estereoscópica. Gracias al cual podían obtener una representación en
tres dimensiones de las trayectorias de las abejas
\citep{art:chiron2013}.

\subsection{\emph{Image Processing for Honey Bee hive Health
Monitoring}}\label{image-processing-for-honey-bee-hive-health-monitoring}

El último artículo publicado data del año 2015 por Rahman Tashakkori y
Ahmad Ghadiri de la \emph{Appalachian State University}. En él, mejoran
el método de detección propuesto en \citep{art:campbell2008} y lo
utilizan para estimar el número de abejas que habrá en un instante de
tiempo dado \citep{art:tashakkori2015}.

\subsection{Comparativa sobre las técnicas utilizadas}\label{comparacion-articulos}

A continuación, se muestran las diferentes técnicas de detección de movimiento,
conteo de abejas y \emph{tracking} utilizadas en los tres artículos 
anteriores y se comparan con las utilizadas en el proyecto.

\tablaSmall{Comparativa métodos de detección de movimiento.}{l l l l l}{comparativa-1}
{ Artículo & Año & Citas & Detección de movimiento \\}{ 
{\citep{art:campbell2008}}   & 2008 & 24 & \emph{Adaptative background subtraction}. \\
{\citep{art:chiron2013}}     & 2013 & 8  & \specialcell{\emph{Adaptative background subtraction}\\\emph{with depth information}.} \\
{\citep{art:tashakkori2015}} & 2015 & 0  & \specialcell{\emph{Averaging a background with}\\\emph{illumination invariant method}.}\\
GoBees                       & 2017 & 0  & \specialcell{\emph{Mixture of Gaussians method}\\(\texttt{BackgroundSubtractorMOG2}).}     \\
} 

\tablaSmall{Comparativa métodos de conteo de abejas y \emph{tracking}.}{l l l}{comparativa-2}
{ Artículo & Conteo de abejas & \emph{Tracking} \\}{ 
{\citep{art:campbell2008}} & \emph{Template-based method.} & \specialcell{\emph{Maximum weighted bipartite}\\\emph{graph matching.}} \\
{\citep{art:chiron2013}} & \specialcell{\emph{Hybrid 3D intensity}\\\emph{depth method.}} & \specialcell{\emph{Kalman filter }y\\\emph{Global Nearest Neighbor}.} \\
{\citep{art:tashakkori2015}} & \emph{Area-based method.} & No \\
GoBees & \emph{Area-based method.} & No \\
} 
\newpage

\section{Proyectos}\label{proyectos}

\subsection{EyesOnHives}\label{eyesonhives}

EyesOnHives es el principal competidor del proyecto. Se trata de un
producto comercial cuyo fin es la monitorización del estado de salud de
las colmenas mediante su actividad de vuelo. Integra un \emph{hardware}
específico que se encarga de la captación de imágenes y una plataforma
en la nube que las procesa y permite el acceso a los datos.

\begin{itemize}
\tightlist
\item
  Web del proyecto: \url{http://www.keltronixinc.com}
\end{itemize}

\subsection{HiveTool}\label{hivetool}

Se trata de un proyecto \emph{OpenSource} que ofrece un conjunto de
herramientas para monitorizar distintos parámetros de una colmena. Una
de estas herramientas es ``Bee Counter'', un contador de abejas por
visión artificial desarrollado sobre una Raspberry Pi.

\begin{itemize}
\tightlist
\item
  Web del proyecto: \url{http://hivetool.org}
\end{itemize}

\section{Fortalezas y debilidades del
proyecto}\label{fortalezas-y-debilidades-del-proyecto}

\begin{table}[H]
\centering
\begin{tabular}{lccc}
\toprule
Características                 & GoBees     & EyesOnHives & HiveTool   \\
\midrule
No requiere \emph{hardware} específico & \cellcolor{green!25} {$\checkmark$} & \cellcolor{red!25} {$\times$} & \cellcolor{red!25} {$\times$} \\
Instalación sencilla            & \cellcolor{green!25} {$\checkmark$} & \cellcolor{green!25} {$\checkmark$}  & \cellcolor{red!25} {$\times$} \\
Procesamiento en local          & \cellcolor{green!25} {$\checkmark$} & \cellcolor{yellow!25} Parcial & \cellcolor{green!25} {$\checkmark$}  \\
No requiere \emph{wifi}         & \cellcolor{green!25} {$\checkmark$} & \cellcolor{red!25} {$\times$} & \cellcolor{green!25} {$\checkmark$}  \\
No requiere red eléctrica       & \cellcolor{green!25} {$\checkmark$} & \cellcolor{red!25} {$\times$}  & \cellcolor{green!25} {$\checkmark$}  \\
Localización GPS                & \cellcolor{green!25} {$\checkmark$} & \cellcolor{red!25} {$\times$} & \cellcolor{red!25} {$\times$}        \\
Gratuito                        & \cellcolor{green!25} {$\checkmark$} & \cellcolor{red!25} {$\times$}  & \cellcolor{green!25} {$\checkmark$}  \\
Plataformas                     & Android    & Web App     & Linux     \\
\bottomrule
\end{tabular}
\caption{Comparativa de las características de los proyectos.}
\label{comparativa-proyectos}
\end{table}

Las principales fortalezas del proyecto son:

\begin{itemize}
\tightlist
\item
  No se necesita adquirir ningún \emph{hardware} específico como en el
  resto de proyectos, simplemente se necesita un \emph{smartphone} con
  Android. Esto hace el proyecto mucho más accesible a los potenciales
  usuarios.
\item
  La instalación es muy sencilla. Únicamente se requiere un trípode o
  cualquier otro tipo de soporte que permita sujetar el
  \emph{smartphone} en posición cenital.
\item
  El procesamiento de las imágenes se realiza en local no en un
  servidor. Considerando que los colmenares suelen estar en medio del
  monte, no podemos requerir una conexión \emph{wifi} como necesita
  EyesOnHives y el envío de vídeo mediante tecnologías 3G/4G supondría
  un coste económico muy elevado.
\item
  No requiere estar conectado a la red eléctrica. El \emph{smartphone}
  cuenta con su propia batería. El consumo de la aplicación no es muy
  elevado al estar la pantalla apagada durante la monitorización. Aun
  así, se pueden utilizar \emph{powerbanks} (baterías portátiles) en
  caso de ser necesarios.
\item
  El \emph{smartphone} tiene integradas varias tecnologías de
  transmisión de información. Lo que da la posibilidad de crear una
  plataforma que centralice la recogida de datos de varios dispositivos
  sin importar su localización.
\item
  Relacionado con el punto anterior, el \emph{smartphone} nos permite
  estar conectados a internet, posibilitándonos ampliar la información
  que maneja nuestra aplicación. Por ejemplo, podemos acceder a la
  información meteorológica en tiempo real.
\item
  El GPS del \emph{smartphone} nos permite localizar geográficamente la
  monitorización y, por tanto, la información meteorológica. Además,
  puede ser de utilidad en caso de robo, gracias a aplicaciones como
  \emph{Android Device Manager}, Cerberus, etc. que permiten localizar
  el dispositivo de forma remota.
\end{itemize}

Las principales debilidades son:

\begin{itemize}
\tightlist
\item
  Actualmente solo se encuentra disponible para Android. Aunque en una
  segunda fase del proyecto se creará una plataforma en la nube que
  centralice todos los datos y una aplicación web que permita acceder a
  ellos.
\item
  El utilizar un \emph{smartphone} como soporte \emph{hardware} tiene
  sus ventajas, pero también sus inconvenientes. La cámara no tiene el
  mismo rendimiento que una cámara diseñada específicamente para esta
  tarea. Esto nos ha limitado en las técnicas de visión artificial que
  hemos podido aplicar, por no disponer de imágenes con la suficiente
  nitidez.
\end{itemize}
